\section{Results} \label{results}
\subsection{Metric definitions}\label{subsec:metric}
For each algorithm $A$, applied on instance $i$, using different randomly generated seeds one have to compute:
\begin{itemize}
  \item  Percentage of runs with constraint violations (0.xxyyy has to be interpreted as xx.yyy\%)
  \item  Mean penalized relative percentage deviation
\end{itemize}

In order to compute these statistics, each algorithm $A$ is launched 25 times on the same instance, measuring the following quantities on each run:
\begin{itemize}
  \item Constraint violations at the end of the run  
  \item Penalized relative percentage deviation of the final solution with respect to the optimal one.
  \item Computation time (CPU time).
\end{itemize}

For each instance $i$, the distributions of PRPD are also displayed using box-plots and tested using the Wilcoxon signed rank statistical test, in order to assess the existence of a statistically significant difference among the results obtained by the different algorithms (ACO and SA) on the same instance. 

\paragraph{Constraint Violations}
In the standard formulation of the TSP problem, a solution to the problem is represented by a permutation of the different
entities (solution components), that the hypothetical traveling salesman has to visit.

The best solution for the problem is the permutation that minimizes the total traveling time (distance) among the cities.
The presence of time windows introduce an additional constraint on the feasibility of the solution.

In fact, each solution component has an associated time window within which it has to be visited in order to guarantee the feasibility of the tour.

Arriving in a (city) before the opening of the corresponding time window involves a delay in the total traveling time (to wait for the time window to open) whereas the arrival after the closure of the time windows will generate a constraint violation.

Thus, a solution is feasible if and only if all the time windows constraints are met, or in other words, if there are no constraint violations.

In this case, the best solution is the feasible solution which minimizes the total travel time.

\paragraph{Penalized Relative Percentage Deviation}
The penalized relative percentage deviation (PRDP from now on) is a measure of the solution quality, with respect to the best known solution for the instance, taking into account a strong penalization for the violation of constraints.

The PRPD is computed as follows:
\begin{equation}
pRPD_{kri} = 100 \cdot \frac{(f_{kri} + 10^4\cdot\Omega_{kri})-best_i}{best_i}
\end{equation}

\paragraph{Run-time}
The run-time is a measure of both the quality and the time complexity of the algorithm.

It is measured using the function \verb|int clock_gettime(clockid_t clk_id, struct timespect *tp)| from the \verb|time.h| library.

The parameter \verb|clk_id=CLOCK_PROCESS_CPUTIME_ID|, is used to read the values from an high-resolution timer provided by the CPU for each process.

The run-time is computed (using the user defined function \verb|ComputeRunTime|) as the difference, with a resolution of $10^-9$ s, from the time obtained using \verb|clock_gettime| at the beginning and the one obtained at the end of the simulation.


\subsection{Run-time distribution}
\subsubsection{Formal definition}
Given an SLS algorithm $A$ and an optimization problem $\Pi$, the success probability of $A$ on a given instance $\pi \in \Pi$ is defined as: 
\begin{equation}
  P_s[RT_{A,\pi} \le t,SQ_{A,\pi} \le q]
\end{equation}

That is, the probability of finding a solution of the problem whose quality is less or equal than $q$ in a time smaller or equal than $t$.

The solution quality $q$ is often expressed as relative solution quality:
\begin{equation}
  q_r = \frac{q}{q_b} - 1
\end{equation}

where $q_b$ is the solution quality of the best known solution for the considered instance $\pi$.

The run-time distribution of $A$ on $\pi$ consists of the probability distribution of the bi-variate random variable ($RT_{A,\pi},SQ_{A,\pi}$).

In other words, the run-time distribution can be defined as: 
\begin{equation}
  RTD_{A,\pi}(t,q) = P_s[RT_{a,\pi} \le t,SQ_{a,\pi} \le q]
\end{equation}

$RTD:\mathbb{R}^{+} \times \mathbb{R}^{+} \mapsto [0,1]$ is indeed a function, defined for algorithm $A$ on instance $\pi \in \Pi$, that maps to each pair run-time $t$, solution quality $q$ the corresponding success probability.


On one hand, by fixing the solution quality to a certain value $q^{*}$ one can obtain the qualified run-time distribution:
\begin{equation}
  QRTD_{A,\pi,q^{*}}(t) = RTD_{A,\pi}(t,q^{*}) = P_s[RT_{a,\pi} \le t,SQ_{a,\pi} \le q^{*}]
\end{equation}

which represents the probability of finding a solution of a better quality than $q^{*}$ as a function of the run-time $t$.


On the other hand, by fixing the maximum computation time to the value $t^{*}$ one obtains the solution quality distribution:
\begin{equation}
  SQD_{A,\pi,t^{*}}(q) = RTD_{A,\pi}(t^{*},q) = P_s[RT_{a,\pi} \le t^{*},SQ_{a,\pi} \le q]
\end{equation}

which determines the probability of finding a solution having a certain quality $q$ given the time bound $t^{*}$.

These are marginal distributions of the run-time one, which allow to have a better insight, respectively, on the time required for the algorithm to find a reasonably good solution and conversely, on the solution quality that can be obtained by fixing a computation time bound, of the algorithm $A$ on instance $\pi$.

\subsubsection{Measurement}\label{subsec:measure}
In the case of the Travel ling Salesman Problem with Time Windows, the algorithm has not only to minimize the total Travel ling time, but also respect all the constraints determined by the time window of each of the nodes.

The solution quality is defined by aggregating the two objectives and taking into account a penalization $p_e = 10^4$ for each constraint violations:
\begin{equation}
  q(p) = f(p) + p_e \cdot\Omega(p)
\end{equation}

from which
\begin{equation}
  q_r(p) = \frac{q(p)}{q_{best}} - 1
\end{equation}

It should be noted that, since the number of constraint violations $\Omega(p)$ constitutes a hard constraint on the feasibility of the solution, the value of the penality $p_e$ is at least ten times bigger than the magnitude of the worst solutions.

This implies that infeasible solutions with a shorter tour with respect to the optimal one, will have an higher relative solution quality $q_r$ than worse but yet feasible solutions.

The measurement of the run-time distributions to reach sufficiently high quality solution, of the implemented algorithms, consists indeed in the estimation of the qualified run-time distribution $QRTD_{A,\pi,q_r^{*}}(t)$ for a sufficiently high relative solution quality $q_r^{*} \in (0,0.025)$.

By running the algorithms for a reasonable number of repetitions the success probability $P_s$ can be approximated by the relative observed frequencies as:
\begin{equation}
P_s[RT_{A,\pi} \le t,RSQ_{A,\pi} \le q_r^{*}] \approx \frac{\text{\#repetitions s.t. } (RT_{A,\pi} \le t \wedge RSQ_{A,\pi} \le q_r^{*})}{\text{\#repetitions}}   
\end{equation}

To compute the relative frequencies, the solution quality will be measured at regular intervals of time in order to have enough samples to ensure a good quality plot.
Since the time scale of the plot will be logarithmic, the sampling process will occur in a way that every interval $[10^i,10^{i+1}]$ will contain the same number $s=50$ of relative solution qualites samples.

\subsection{Results} 

% Measure, for each of the implemented algorithms on 5 instances, the run-time distributions to reach suf-
% ficiently high quality solutions (e.g. best-known solutions available at http://iridia.ulb.ac.be/
%  ̃manuel/tsptw-instances#instances).
% Measure the run-time distributions across 25 repetitions using a cut-off time of 10 times the termination
% criterion above.

% In addition to the randomly generated initial function, I developed the \verb|GenerateHeuristicInitialSolution()|
% which construct the initial solution according to an heuristic which aims to minimize the number of constraint violations.

% The general outline of the algorithm is as follows:
% \begin{enumerate}
%   \item Construct a solution by ordering the solution components in an ascending way according to their time windows closing time
%   \item For a number of times equal to the solution size divided by 10:
%   \begin{enumerate}
%     \item Select a random solution component in the solution
%     \item Select a neighborhood size randomly.
%     \item Shuffle the elements in the chosen neighborhood of the chosen solution
%   \end{enumerate}
% \end{enumerate}

% Due to the lack of time, I was only able to test the heuristic function on a limited number of instances.
% The same metrics as in \ref{subsec:metric} will be used to evaluate the algorithms.

% \subsubsection{n80w200.001}
% \begin{center}
% \includegraphics[width=0.6\textwidth,keepaspectratio]{{II-H/n80w200.001-CpuTime}.pdf}
% \captionof{figure}{n80w200.001 - Runtime boxplots for the different iterative improvement algorithms with heuristic initialization}
% \end{center}

% \begin{center}
% \includegraphics[width=0.6\textwidth,keepaspectratio]{{II-H/n80w200.001-PRPD}.pdf}
% \captionof{figure}{n80w200.001 - PRPD boxplots for the different iterative improvement algorithms with heuristic initialization}
% \end{center}

% \begin{center}
% \begin{tabular}{|l|l|}
% \hline
% \textbf{Test} & \textbf{P-Value} \\
% \hline
% First vs best - Transpose&3.95591160889952e-18\\
% \hline
% First vs best - Exchange&3.95591160889952e-18\\
% \hline
% First vs best - Insert&7.16468868599392e-06\\
% \hline
% Exchange vs Insert - First&3.95591160889952e-18\\
% \hline
% Exchange vs Insert - Best&3.9550194074242e-18\\
% \hline
% \end{tabular}
% \captionof{table}{n80w200.001 - Results of Wilcoxon paired signed rank test}
% \label{tab:w.21}
% \end{center}

% \subsubsection{n80w200.002}
% \begin{center}
% \includegraphics[width=0.6\textwidth,keepaspectratio]{{II-H/n80w200.002-CpuTime}.pdf}
% \captionof{figure}{n80w200.002 - Runtime boxplots for the different iterative improvement algorithms with heuristic initialization}
% \end{center}

% \begin{center}
% \includegraphics[width=0.6\textwidth,keepaspectratio]{{II-H/n80w200.002-PRPD}.pdf}
% \captionof{figure}{n80w200.002 - PRPD boxplots for the different  iterative improvement algorithms with heuristic initialization}
% \end{center}

% \begin{center}
% \begin{tabular}{|l|l|}
% \hline
% \textbf{Test} & \textbf{P-Value} \\
% \hline
% First vs best - Transpose&3.95591160889952e-18\\
% \hline
% First vs best - Exchange&3.95591160889952e-18\\
% \hline
% First vs best - Insert&1.5011633635878e-17\\
% \hline
% Exchange vs Insert - First&3.95591160889952e-18\\
% \hline
% Exchange vs Insert - Best&3.9552424399092e-18\\
% \hline
% \end{tabular}
% \captionof{table}{n80w200.002 - Results of Wilcoxon paired signed rank test}
% \label{tab:w.22}
% \end{center}

% \subsubsection{n80w200.003}
% \begin{center}
% \includegraphics[width=0.6\textwidth,keepaspectratio]{{II-H/n80w200.003-CpuTime}.pdf}
% \captionof{figure}{n80w200.003 - Runtime boxplots for the different iterative improvement algorithms with heuristic initialization}
% \end{center}

% \begin{center}
% \includegraphics[width=0.6\textwidth,keepaspectratio]{{II-H/n80w200.003-PRPD}.pdf}
% \captionof{figure}{n80w200.003 - PRPD boxplots for the different iterative improvement algorithms with heuristic initialization}
% \end{center}

% \begin{center}
% \begin{tabular}{|l|l|}
% \hline
% \textbf{Test} & \textbf{P-Value} \\
% \hline
% First vs best - Transpose&3.95591160889952e-18\\
% \hline
% First vs best - Exchange&3.95591160889952e-18\\
% \hline
% First vs best - Insert&2.88431649979563e-16\\
% \hline
% Exchange vs Insert - First&3.9556885406462e-18\\
% \hline
% Exchange vs Insert - Best&4.33074349739998e-18\\
% \hline
% \end{tabular}
% \captionof{table}{n80w200.003 - Results of Wilcoxon paired signed rank test}
% \label{tab:w.23}
% \end{center}

% \subsection{Statistics}

% \subsubsection{Transpose-First Improvement}
% \begin{center}
% \begin{tabular}{|l|c|l|l|}
% \hline
% \textbf{Instance}& \textbf{\% Infeasible} & $\mathbf{\bar{PRDP}}$ &$\mathbf{\bar{Runtime}}$\\
% \hline
% n80w200.003&1&1268365.4&0.0065871265\\
% \hline
% n80w200.002&1&1071407.7&0.007598828\\
% \hline
% n80w200.001&1&1243295.2&0.0093113496\\
% \hline
% \end{tabular}
% \captionof{table}{Statistics summary for iterative improvement algorithm with Transpose neighborhood and First Improvement pivoting rule}
% \label{tab:t.f.h}
% \end{center}

% \subsubsection{Transpose-Best Improvement}
% \begin{center}
% \begin{tabular}{|l|c|l|l|}
% \hline
% \textbf{Instance}& \textbf{\% Infeasible} & $\mathbf{\bar{PRDP}}$ &$\mathbf{\bar{Runtime}}$\\
% \hline
% n80w200.003&1&1266853.5&0.0106505552\\
% \hline
% n80w200.002&1&1071417.8&0.011013054\\
% \hline
% n80w200.001&1&1240638.1&0.013889943\\
% \hline
% \end{tabular}
% \captionof{table}{Statistics summary for iterative improvement algorithm with Transpose neighborhood and Best Improvement pivoting rule}
% \label{tab:t.b.h}
% \end{center}

% \subsubsection{Exchange-First Improvement}
% \begin{center}
% \begin{tabular}{|l|c|l|l|}
% \hline
% \textbf{Instance}& \textbf{\% Infeasible} & $\mathbf{\bar{PRDP}}$ &$\mathbf{\bar{Runtime}}$\\
% \hline
% n80w200.003&0&27.068636&6.0144574\\
% \hline
% n80w200.002&0.02&432.979367&6.1109123\\
% \hline
% n80w200.001&0.87&18146.162472&8.842327\\
% \hline
% \end{tabular}
% \captionof{table}{Statistics summary for iterative improvement algorithm with Exchange neighborhood and First Improvement pivoting rule}
% \label{tab:e.f.h}
% \end{center}

% \subsubsection{Exchange-Best Improvement}
% \begin{center}
% \begin{tabular}{|l|c|l|l|}
% \hline
% \textbf{Instance}& \textbf{\% Infeasible} & $\mathbf{\bar{PRDP}}$ &$\mathbf{\bar{Runtime}}$\\
% \hline
% n80w200.003&1&180715.273&17.098637\\
% \hline
% n80w200.002&1&92647.622&12.3893899\\
% \hline
% n80w200.001&1&694332.77&17.575257\\
% \hline
% \end{tabular}
% \captionof{table}{Statistics summary for iterative improvement algorithm with Exchange neighborhood and Best Improvement pivoting rule}
% \label{tab:e.b.h}
% \end{center}

% \subsubsection{Insert-First Improvement}
% \begin{center}
% \begin{tabular}{|l|c|l|l|}
% \hline
% \textbf{Instance}& \textbf{\% Infeasible} & $\mathbf{\bar{PRDP}}$ &$\mathbf{\bar{Runtime}}$\\
% \hline
% n80w200.003&0.03&2587.6121429&30.841123\\
% \hline
% n80w200.002&0&9.9979528&35.672697\\
% \hline
% n80w200.001&0.1&4897.5213208&32.775532\\
% \hline
% \end{tabular}
% \captionof{table}{Statistics summary for iterative improvement algorithm with Insert neighborhood and First Improvement pivoting rule}
% \label{tab:i.f.h}
% \end{center}

% \subsubsection{Insert-Best Improvement}
% \begin{center}
% \begin{tabular}{|l|c|l|l|}
% \hline
% n80w200.003&0&4.3454898&23.452399\\
% \hline
% n80w200.002&0.01&630.9635938&27.922795\\
% \hline
% n80w200.001&0.1&3680.5235577&32.277408\\
% \hline
% \end{tabular}
% \captionof{table}{Statistics summary for iterative improvement algorithm with Insert neighborhood and Best Improvement pivoting rule}
% \label{tab:i.b.h}
% \end{center}

% \subsection{Results discussion}
% By looking at tables \ref{tab:t.f.h}, \ref{tab:t.b.h}, \ref{tab:e.f.h}, \ref{tab:e.b.h} \ref{tab:i.f.h}, \ref{tab:i.b.h} on can see that:
% \begin{itemize}

% \item The only neighborhood type which does not allow to generate feasible solution is the Transpose one.

% \item By using the heuristic, also the algorithm using the Exchange neighbohood is able to generate feasible solutions that are close to the best known value.

% \item The use of the heuristic allows for a consistent reduction of the runtime, which becomes on average on half ot the runtime of the algorithm with random initiaalisation on the same instances.

% \item The solution quality of the generated solutions also benefits from the introduction of an heurstic initialization.

% \item Tables \ref{tab:w.21}, \ref{tab:w.22}, \ref{tab:w.23} contain, in any case, p-values considerably smaller than the significance level ($\alpha=0.05$). 

% This implies that the null hypothesis corresponding to the equality of the median values of the differences of the two distributions can be rejected, hence assessing the existence of a statistically significant difference among the solution quality generated by analyzed algorithms.


% \end{itemize}

% \subsection{Experiment results}
% \subsubsection{n80w20.001}
% \begin{center}
% \includegraphics[width=0.6\textwidth,keepaspectratio]{{II/n80w20.001/n80w20.001-CpuTime}.pdf}
% \captionof{figure}{n80w20.001 - Runtime boxplots for the different iterative improvement algorithms}
% \end{center}

% \begin{center}
% \includegraphics[width=0.6\textwidth,keepaspectratio]{{II/n80w20.001/n80w20.001-PRPD}.pdf}
% \captionof{figure}{n80w20.001 - PRPD boxplots for the different iterative improvement algorithms}
% \end{center}

% \begin{center}
% \begin{tabular}{|l|l|}
% \hline
% \textbf{Test} & \textbf{P-Value} \\
% \hline
% First vs best - Transpose&9.74631639820544e-18\\
% \hline
% First vs best - Exchange&2.04966732989559e-17\\
% \hline
% First vs best - Insert&1.74838327736385e-15\\
% \hline
% Exchange vs Insert - First&3.95591160889952e-18\\
% \hline
% Exchange vs Insert - Best&3.9556885406462e-18\\
% \hline
% \end{tabular}
% \captionof{table}{n80w20.001 - Results of Wilcoxon paired signed rank test}
% \label{tab:w.1}
% \end{center}

% \subsubsection{n80w20.002}
% \begin{center}
% \includegraphics[width=0.6\textwidth,keepaspectratio]{{II/n80w20.002/n80w20.002-CpuTime}.pdf}
% \captionof{figure}{n80w20.002 - Runtime boxplots for the different iterative improvement algorithms}
% \end{center}

% \begin{center}
% \includegraphics[width=0.6\textwidth,keepaspectratio]{{II/n80w20.002/n80w20.002-PRPD}.pdf}
% \captionof{figure}{n80w20.002 - PRPD boxplots for the different iterative improvement algorithms}
% \end{center}

% \begin{center}
% \begin{tabular}{|l|l|}
% \hline
% \textbf{Test} & \textbf{P-Value} \\
% \hline
% First vs best - Transpose&3.95591160889952e-18\\
% \hline
% First vs best - Exchange&1.61703099974578e-17\\
% \hline
% First vs best - Insert&2.39050570998277e-07\\
% \hline
% Exchange vs Insert - First&3.95591160889952e-18\\
% \hline
% Exchange vs Insert - Best&3.9556885406462e-18\\
% \hline
% \end{tabular}
% \captionof{table}{n80w20.002 - Results of Wilcoxon paired signed rank test}
% \label{tab:w.2}
% \end{center}

% \subsubsection{n80w20.003}
% \begin{center}
% \includegraphics[width=0.6\textwidth,keepaspectratio]{{II/n80w20.003/n80w20.003-CpuTime}.pdf}
% \captionof{figure}{n80w20.003 - Runtime boxplots for the different iterative improvement algorithms}
% \end{center}

% \begin{center}
% \includegraphics[width=0.6\textwidth,keepaspectratio]{{II/n80w20.003/n80w20.003-PRPD}.pdf}
% \captionof{figure}{n80w20.003 - PRPD boxplots for the different iterative improvement algorithms}
% \end{center}

% \begin{center}
% \begin{tabular}{|l|l|}
% \hline
% \textbf{Test} & \textbf{P-Value} \\
% \hline
% First vs best - Transpose&3.95591160889952e-18\\
% \hline
% First vs best - Exchange&6.21747363653032e-18\\
% \hline
% First vs best - Insert&6.2952945764779e-08\\
% \hline
% Exchange vs Insert - First&3.9556885406462e-18\\
% \hline
% Exchange vs Insert - Best&3.95591160889952e-18\\
% \hline
% \end{tabular}
% \captionof{table}{n80w20.003 - Results of Wilcoxon paired signed rank test}
% \label{tab:w.3}
% \end{center}

% \subsubsection{n80w20.004}
% \begin{center}
% \includegraphics[width=0.6\textwidth,keepaspectratio]{{II/n80w20.004/n80w20.004-CpuTime}.pdf}
% \captionof{figure}{n80w20.004 - Runtime boxplots for the different iterative improvement algorithms}
% \end{center}

% \begin{center}
% \includegraphics[width=0.6\textwidth,keepaspectratio]{{II/n80w20.004/n80w20.004-PRPD}.pdf}
% \captionof{figure}{n80w20.001 - PRPD boxplots for the different iterative improvement algorithms}
% \end{center}

% \begin{center}
% \begin{tabular}{|l|l|}
% \hline
% \textbf{Test} & \textbf{P-Value} \\
% \hline
% First vs best - Transpose&4.33123080260219e-18\\
% \hline
% First vs best - Exchange&1.5356610755813e-16\\
% \hline
% First vs best - Insert&4.27702026764362e-14\\
% \hline
% Exchange vs Insert - First&5.59593516960623e-18\\
% \hline
% Exchange vs Insert - Best&3.95591160889952e-18\\
% \hline
% \end{tabular}
% \captionof{table}{n80w20.004 - Results of Wilcoxon paired signed rank test}
% \label{tab:w.4}
% \end{center}

% \subsubsection{n80w20.005}
% \begin{center}
% \includegraphics[width=0.6\textwidth,keepaspectratio]{{II/n80w20.005/n80w20.005-CpuTime}.pdf}
% \captionof{figure}{n80w20.005 - Runtime boxplots for the different iterative improvement algorithms}
% \end{center}

% \begin{center}
% \includegraphics[width=0.6\textwidth,keepaspectratio]{{II/n80w20.005/n80w20.005-PRPD}.pdf}
% \captionof{figure}{n80w20.005 - PRPD boxplots for the different iterative improvement algorithms}
% \end{center}

% \begin{center}
% \begin{tabular}{|l|l|}
% \hline
% \textbf{Test} & \textbf{P-Value} \\
% \hline
% First vs best - Transpose&4.46398542390809e-18\\
% \hline
% First vs best - Exchange&4.74166029806301e-18\\
% \hline
% First vs best - Insert&4.0369131744045e-10\\
% \hline
% Exchange vs Insert - First&4.74166029806301e-18\\
% \hline
% Exchange vs Insert - Best&3.95591160889952e-18\\
% \hline
% \end{tabular}
% \captionof{table}{n80w20.005 - Results of Wilcoxon paired signed rank test}
% \label{tab:w.5}
% \end{center}

% \subsubsection{n80w200.001}
% \begin{center}
% \includegraphics[width=0.6\textwidth,keepaspectratio]{{II/n80w200.001/n80w200.001-CpuTime}.pdf}
% \captionof{figure}{n80w200.001 - Runtime boxplots for the different iterative improvement algorithms}
% \end{center}

% \begin{center}
% \includegraphics[width=0.6\textwidth,keepaspectratio]{{II/n80w200.001/n80w200.001-PRPD}.pdf}
% \captionof{figure}{n80w200.001 - PRPD boxplots for the different iterative improvement algorithms}
% \end{center}

% \begin{center}
% \begin{tabular}{|l|l|}
% \hline
% \textbf{Test} & \textbf{P-Value} \\
% \hline
% First vs best - Transpose&4.07730530936212e-18\\
% \hline
% First vs best - Exchange&2.17457280454137e-17\\
% \hline
% First vs best - Insert&3.95591160889952e-18\\
% \hline
% Exchange vs Insert - First&3.95591160889952e-18\\
% \hline
% Exchange vs Insert - Best&3.95591160889952e-18\\
% \hline
% \end{tabular}
% \captionof{table}{n80w200.001 - Results of Wilcoxon paired signed rank test}
% \label{tab:w.6}
% \end{center}

% \subsubsection{n80w200.002}
% \begin{center}
% \includegraphics[width=0.6\textwidth,keepaspectratio]{{II/n80w200.002/n80w200.002-CpuTime}.pdf}
% \captionof{figure}{n80w200.002 - Runtime boxplots for the different iterative improvement algorithms}
% \end{center}

% \begin{center}
% \includegraphics[width=0.6\textwidth,keepaspectratio]{{II/n80w200.002/n80w200.002-PRPD}.pdf}
% \captionof{figure}{n80w200.002 - PRPD boxplots for the different iterative improvement algorithms}
% \end{center}

% \begin{center}
% \begin{tabular}{|l|l|}
% \hline
% \textbf{Test} & \textbf{P-Value} \\
% \hline
% First vs best - Transpose&5.19043683699158e-18\\
% \hline
% First vs best - Exchange&4.6720416035814e-17\\
% \hline
% First vs best - Insert&3.95591160889952e-18\\
% \hline
% Exchange vs Insert - First&3.95591160889952e-18\\
% \hline
% Exchange vs Insert - Best&3.95591160889952e-18\\
% \hline
% \end{tabular}
% \captionof{table}{n80w200.002 - Results of Wilcoxon paired signed rank test}
% \label{tab:w.7}
% \end{center}

% \subsubsection{n80w200.003}
% \begin{center}
% \includegraphics[width=0.6\textwidth,keepaspectratio]{{II/n80w200.003/n80w200.003-CpuTime}.pdf}
% \captionof{figure}{n80w200.003 - Runtime boxplots for the different iterative improvement algorithms}
% \end{center}

% \begin{center}
% \includegraphics[width=0.6\textwidth,keepaspectratio]{{II/n80w200.003/n80w200.003-PRPD}.pdf}
% \captionof{figure}{n80w200.003 - PRPD boxplots for the different iterative improvement algorithms}
% \end{center}

% \begin{center}
% \begin{tabular}{|l|l|}
% \hline
% \textbf{Test} & \textbf{P-Value} \\
% \hline
% First vs best - Transpose&4.33123080260219e-18\\
% \hline
% First vs best - Exchange&7.01070639830382e-18\\
% \hline
% First vs best - Insert&3.95591160889952e-18\\
% \hline
% Exchange vs Insert - First&3.95591160889952e-18\\
% \hline
% Exchange vs Insert - Best&3.95591160889952e-18\\
% \hline
% \end{tabular}
% \captionof{table}{n80w200.003 - Results of Wilcoxon paired signed rank test}
% \label{tab:w.8}
% \end{center}

% \subsubsection{n80w200.004}
% \begin{center}
% \includegraphics[width=0.6\textwidth,keepaspectratio]{{II/n80w200.004/n80w200.004-CpuTime}.pdf}
% \captionof{figure}{n80w200.004 - Runtime boxplots for the different iterative improvement algorithms}
% \end{center}

% \begin{center}
% \includegraphics[width=0.6\textwidth,keepaspectratio]{{II/n80w200.004/n80w200.004-PRPD}.pdf}
% \captionof{figure}{n80w200.001 - PRPD boxplots for the different iterative improvement algorithms}
% \end{center}

% \begin{center}
% \begin{tabular}{|l|l|}
% \hline
% \textbf{Test} & \textbf{P-Value} \\
% \hline
% First vs best - Transpose&4.33123080260219e-18\\
% \hline
% First vs best - Exchange&2.4473398426105e-17\\
% \hline
% First vs best - Insert&3.95591160889952e-18\\
% \hline
% Exchange vs Insert - First&3.95591160889952e-18\\
% \hline
% Exchange vs Insert - Best&3.95591160889952e-18\\
% \hline
% \end{tabular}
% \captionof{table}{n80w200.004 - Results of Wilcoxon paired signed rank test}
% \label{tab:w.9}
% \end{center}

% \subsubsection{n80w200.005}
% \begin{center}
% \includegraphics[width=0.6\textwidth,keepaspectratio]{{II/n80w200.005/n80w200.005-CpuTime}.pdf}
% \captionof{figure}{n80w200.005 - Runtime boxplots for the different iterative improvement algorithms}
% \end{center}

% \begin{center}
% \includegraphics[width=0.6\textwidth,keepaspectratio]{{II/n80w200.005/n80w200.005-PRPD}.pdf}
% \captionof{figure}{n80w200.005 - PRPD boxplots for the different iterative improvement algorithms}
% \end{center}

% \begin{center}
% \begin{tabular}{|l|l|}
% \hline
% \textbf{Test} & \textbf{P-Value} \\
% \hline
% First vs best - Transpose&4.74166029806301e-18\\
% \hline
% First vs best - Exchange&1.40854365025687e-16\\
% \hline
% First vs best - Insert&3.95591160889952e-18\\
% \hline
% Exchange vs Insert - First&3.95591160889952e-18\\
% \hline
% Exchange vs Insert - Best&3.95591160889952e-18\\
% \hline
% \end{tabular}
% \captionof{table}{n80w200.005 - Results of Wilcoxon paired signed rank test}
% \label{tab:w.10}
% \end{center}

% \subsection{Statistics}

% \subsubsection{Transpose-First Improvement}
% \begin{center}
% \begin{tabular}{|l|c|l|l|}
% \hline
% \textbf{Instance}& \textbf{\% Infeasible} & $\mathbf{\bar{PRDP}}$ &$\mathbf{\bar{Runtime}}$\\
% \hline
% n80w20.001&1&1229712.6&0.0100035563\\
% \hline
% n80w20.002&1&1028075.17&0.0095843375\\
% \hline
% n80w20.003&1&1132968.5&0.0098099452\\
% \hline
% n80w20.005&1&1010681.79&0.0097989762\\
% \hline
% n80w20.004&1&1226174.6&0.0094877067\\
% \hline
% n80w200.001&1&1467634.1&0.0098152878\\
% \hline
% n80w200.002&1&1504523.7&0.0099101404\\
% \hline
% n80w200.004&1&1388037.6&0.0098893615\\
% \hline
% n80w200.003&1&1567210.5&0.0097697727\\
% \hline
% n80w200.005&1&1644110.9&0.0097550971\\
% \hline
% \end{tabular}
% \captionof{table}{Statistics summary for iterative improvement algorithm with Transpose neighborhood and First Improvement pivoting rule}
% \label{tab:t.f}
% \end{center}

% \subsubsection{Transpose-Best Improvement}
% \begin{center}
% \begin{tabular}{|l|c|l|l|}
% \hline
% \textbf{Instance}& \textbf{\% Infeasible} & $\mathbf{\bar{PRDP}}$ &$\mathbf{\bar{Runtime}}$\\
% \hline
% n80w20.001&1&1236039.3&0.014545497\\
% \hline
% n80w20.002&1&1033090.48&0.0145450422\\
% \hline
% n80w20.003&1&1137758.7&0.014536776\\
% \hline
% n80w20.005&1&1014818.46&0.014947609\\
% \hline
% n80w20.004&1&1232996&0.015151286\\
% \hline
% n80w200.001&1&1476994.8&0.0147534638\\
% \hline
% n80w200.002&1&1511281&0.014884921\\
% \hline
% n80w200.004&1&1392023.3&0.0146445173\\
% \hline
% n80w200.003&1&1575355.5&0.0143582468\\
% \hline
% n80w200.005&1&1653419.6&0.0150098656\\
% \hline
% \end{tabular}
% \captionof{table}{Statistics summary for iterative improvement algorithm with Transpose neighborhood and Best Improvement pivoting rule}
% \label{tab:t.b}
% \end{center}

% \subsubsection{Exchange-First Improvement}
% \begin{center}
% \begin{tabular}{|l|c|l|l|}
% \hline
% \textbf{Instance}& \textbf{\% Infeasible} & $\mathbf{\bar{PRDP}}$ &$\mathbf{\bar{Runtime}}$\\
% \hline
% n80w20.001&1&1035718.78&18.390814\\
% \hline
% n80w20.002&1&884386.3&18.198632\\
% \hline
% n80w20.003&1&956518.77&18.913801\\
% \hline
% n80w20.005&1&849054.52&19.043496\\
% \hline
% n80w20.004&1&1030411.56&19.660314\\
% \hline
% n80w200.001&1&661322.18&23.117446\\
% \hline
% n80w200.002&1&813339.6&22.552976\\
% \hline
% n80w200.004&1&679489.06&22.068859\\
% \hline
% n80w200.003&1&697450.03&24.338257\\
% \hline
% n80w200.005&1&760027.65&22.348975\\
% \hline
% \end{tabular}
% \captionof{table}{Statistics summary for iterative improvement algorithm with Exchange neighborhood and First Improvement pivoting rule}
% \label{tab:e.f}
% \end{center}

% \subsubsection{Exchange-Best Improvement}
% \begin{center}
% \begin{tabular}{|l|c|l|l|}
% \hline
% \textbf{Instance}& \textbf{\% Infeasible} & $\mathbf{\bar{PRDP}}$ &$\mathbf{\bar{Runtime}}$\\
% \hline
% n80w20.001&1&1086217.68&13.091453\\
% \hline
% n80w20.002&1&901762.05&13.365301\\
% \hline
% n80w20.003&1&1017243.86&13.267128\\
% \hline
% n80w20.005&1&895720.53&13.5824245\\
% \hline
% n80w20.004&1&1084080.4&14.189582\\
% \hline
% n80w200.001&1&1022433.76&16.418229\\
% \hline
% n80w200.002&1&1075649.94&16.028712\\
% \hline
% n80w200.004&1&994703.67&15.518766\\
% \hline
% n80w200.003&1&1094460.9&16.16696\\
% \hline
% n80w200.005&1&1110949.86&16.566802\\
% \hline
% \end{tabular}
% \captionof{table}{Statistics summary for iterative improvement algorithm with Exchange neighborhood and Best Improvement pivoting rule}
% \label{tab:e.b}
% \end{center}

% \subsubsection{Insert-First Improvement}
% \begin{center}
% \begin{tabular}{|l|c|l|l|}
% \hline
% \textbf{Instance}& \textbf{\% Infeasible} & $\mathbf{\bar{PRDP}}$ &$\mathbf{\bar{Runtime}}$\\
% \hline
% n80w20.001&0.65&16070.27322082&25.88279\\
% \hline
% n80w20.002&0.83&11803.71462682&28.509938\\
% \hline
% n80w20.003&0.83&21587.617&28.579453\\
% \hline
% n80w20.005&0.32&4945.642107&28.672083\\
% \hline
% n80w20.004&0.49&10894.01867489&28.474429\\
% \hline
% n80w200.001&0.22&6119.9927045&59.544212\\
% \hline
% n80w200.002&0&11.2561521&64.580306\\
% \hline
% n80w200.004&0.11&3049.80805246&63.942238\\
% \hline
% n80w200.003&0.01&437.87774695&63.687806\\
% \hline
% n80w200.005&0.01&464.62990671&66.084369\\
% \hline
% \end{tabular}
% \captionof{table}{Statistics summary for iterative improvement algorithm with Insert neighborhood and First Improvement pivoting rule}
% \label{tab:i.f}
% \end{center}

% \subsubsection{Insert-Best Improvement}
% \begin{center}
% \begin{tabular}{|l|c|l|l|}
% \hline
% \textbf{Instance}& \textbf{\% Infeasible} & $\mathbf{\bar{PRDP}}$ &$\mathbf{\bar{Runtime}}$\\
% \hline
% n80w20.001&0.56&14772.44442862&28.769951\\
% \hline
% n80w20.002&0.84&16009.51797952&30.159812\\
% \hline
% n80w20.003&0.89&26984.256&30.197019\\
% \hline
% n80w20.005&0.52&10159.3062354&30.776906\\
% \hline
% n80w20.004&0.59&18861.32619536&31.013\\
% \hline
% n80w200.001&0.37&21195.2243605&33.031965\\
% \hline
% n80w200.002&0&13.2172159&33.011011\\
% \hline
% n80w200.004&0.42&23019.4946845&34.123681\\
% \hline
% n80w200.003&0.12&7740.06242146&34.080592\\
% \hline
% n80w200.005&0.07&2516.0079072&33.450537\\
% \hline
% \end{tabular}
% \captionof{table}{Statistics summary for iterative improvement algorithm with Insert neighborhood and Best Improvement pivoting rule}
% \label{tab:i.b}
% \end{center}

% \subsection{Results discussion}
% By looking at tables \ref{tab:t.f}, \ref{tab:t.b}, \ref{tab:e.f}, \ref{tab:e.b} \ref{tab:i.f}, \ref{tab:i.b} on can see that:
% \begin{itemize}
% \item The neighborhood type has a strong influence on the both the time complexity of the algorithm and the generated solution quality. This is due to the size of the different neighborhoods ($n=80$):
%       \begin{itemize}
%         \item Transpose - $(n-1)$
%         \item Exchange - $\frac{n\cdot(n-1)}{2}$
%         \item Insert - $(n-1)^2$
%       \end{itemize}
% The different size of the neighborhoods corresponds to different degrees of exploration (diversification).
      
% \item Transpose and Exchange neighborhoods have smaller runtimes but a percentage of infeasible runs equal to 1. 
% Both the algorithm do not allow to find a feasible solution but the Exchange algorithm constructs solutions with a better quality (reduced, but not yet null, constraint violations and total travel time).

% \item Insert is the only neighborhood type that allows to generate solutions that are both feasible and closer to the global optima.

% \item The first-improvement pivoting rule is generally slower than the best-improvement one, when considering the same neighborhood type.
% This is due to the fact that, with the first-improvement pivoting rule, smaller improvement are made to the solution at each iteration, thus requiring and higher number of iteration to converge to a local optima, with respect to the case where the best improvement is chosen at each time step.

% \item The quality of the solutions generated using the first-improvement pivoting rule is slightly better thant those generated using the best-improvement one.

% \item Tables \ref{tab:w.1}, \ref{tab:w.2}, \ref{tab:w.3}, \ref{tab:w.4}, \ref{tab:w.5}, \ref{tab:w.6}, \ref{tab:w.7}, \ref{tab:w.8}, \ref{tab:w.9}, \ref{tab:w.10} contain, in any case, p-values considerably smaller than the significance level ($\alpha=0.05$). 

% This implies that the null hypothesis corresponding to the equality of the median values of the differences of the two distributions can be rejected, hence assessing the existence of a statistically significant difference among the solution quality generated by analyzed algorithms.

% \item By looking at the Cpu time, one can easily see that the instances \emph{n80w20.X} have lower runtimes than the \emph{n80w200.X} ones. They can then be considered, with respect to the iterative improvement algorithms, simpler instances with respect to the latter.

% \end{itemize}

% An Iterative Improvement algorithm generally starts from a candidate solution, which can be either generated randomly or using an heuristic, and improves the evaluation of the solution at each step by modifying the solution structure , until a local optimum is reached.

% In the previous problem, I considered different kinds of 2-opt neighborhood, and different pivoting rules.

% This means that, at each step, a new solution is constructed from the current best by modifying only two solution components (with Transpose,Exchange or Insert operations) and only the first/best improving solution will become the new best soluion.

% The main limitation of such kind of algorithms is that they tend to get stuck in solutions that are locally optimum but not globally.

% Provided that:
% \begin{itemize}
% \item A global optimum is optimal with respect to any kind of neighborhood. 
% \item A solution that is locally optimal with respect to a neighborhood may not be optimal with respect to other kinds of neighborhood
% \end{itemize}
% by dynamically changing the neighborhood type an algorithm is able to escape local optima.

% This section will analyse the results of the execution of two variable neighborhood descent algorithm, based on the previously analyzed iterative improvement algorithms :
% \begin{itemize}
% \item \textbf{Standard Variable Neighborhood Descent} (i.e. Changing neighborhood when a local optimum is encoutered, until the neighborhood chain is terminated and going back to the smallest neighborhood every time the local optimum is escaped.)
% \item \textbf{Piped Variable Neighborhood Descent} (i.e. Using the locally optimum solution found using one neighborhood type in the chain as the initial solution for the following type.)
% \end{itemize}
% The same metrics as in \ref{subsec:metric} will be used to evaluate the algorithms.

% \subsection{Experiment results}
% \subsubsection{n80w20.001}
% \begin{center}
% \includegraphics[width=0.6\textwidth,keepaspectratio]{{VND/n80w20.001/n80w20.001-CpuTime}.pdf}
% \captionof{figure}{n80w20.001 - Runtime boxplots for the different variable neighborhood descent algorithms}
% \end{center}

% \begin{center}
% \includegraphics[width=0.6\textwidth,keepaspectratio]{{VND/n80w20.001/n80w20.001-PRPD}.pdf}
% \captionof{figure}{n80w20.001 - PRPD boxplots for the different variable neighborhood descent algorithms}
% \end{center}

% \begin{center}
% \begin{tabular}{|l|l|}
% \hline
% \textbf{Test} & \textbf{P-Value} \\
% \hline
% Tei vs Tie - Standard&3.95591160889952e-18\\
% \hline
% Tei vs Tie - Piped&3.9556885406462e-18\\
% \hline
% Standard vs Piped - Tei&3.95591160889952e-18\\
% \hline
% Standard vs Piped - Tie&3.95591160889952e-18\\
% \hline
% \end{tabular}
% \captionof{table}{n80w20.001 - Results of Wilcoxon paired signed rank test}
% \label{tab:w.11}
% \end{center}

% \subsubsection{n80w20.002}
% \begin{center}
% \includegraphics[width=0.6\textwidth,keepaspectratio]{{VND/n80w20.002/n80w20.002-CpuTime}.pdf}
% \captionof{figure}{n80w20.002 - Runtime boxplots for the different variable neighborhood descent algorithms}
% \end{center}

% \begin{center}
% \includegraphics[width=0.6\textwidth,keepaspectratio]{{VND/n80w20.002/n80w20.002-PRPD}.pdf}
% \captionof{figure}{n80w20.002 - PRPD boxplots for the different variable neighborhood descent algorithms}
% \end{center}

% \begin{center}
% \begin{tabular}{|l|l|}
% \hline
% \textbf{Test} & \textbf{P-Value} \\
% \hline
% Tei vs Tie - Standard&3.9556885406462e-18\\
% \hline
% Tei vs Tie - Piped&3.95591160889952e-18\\
% \hline
% Standard vs Piped - Tei&3.95591160889952e-18\\
% \hline
% Standard vs Piped - Tie&3.95591160889952e-18\\
% \hline
% \end{tabular}
% \captionof{table}{n80w20.002 - Results of Wilcoxon paired signed rank test}
% \label{tab:w.12}
% \end{center}

% \subsubsection{n80w20.003}
% \begin{center}
% \includegraphics[width=0.6\textwidth,keepaspectratio]{{VND/n80w20.003/n80w20.003-CpuTime}.pdf}
% \captionof{figure}{n80w20.003 - Runtime boxplots for the different variable neighborhood descent algorithms}
% \end{center}

% \begin{center}
% \includegraphics[width=0.6\textwidth,keepaspectratio]{{VND/n80w20.003/n80w20.003-PRPD}.pdf}
% \captionof{figure}{n80w20.003 - PRPD boxplots for the different variable neighborhood descent algorithms}
% \end{center}

% \begin{center}
% \begin{tabular}{|l|l|}
% \hline
% \textbf{Test} & \textbf{P-Value} \\
% \hline
% Tei vs Tie - Standard&3.9552424399092e-18\\
% \hline
% Tei vs Tie - Piped&3.95591160889952e-18\\
% \hline
% Standard vs Piped - Tei&3.95591160889952e-18\\
% \hline
% Standard vs Piped - Tie&3.95591160889952e-18\\
% \hline
% \end{tabular}
% \captionof{table}{n80w20.003 - Results of Wilcoxon paired signed rank test}
% \label{tab:w.13}
% \end{center}

% \subsubsection{n80w20.004}
% \begin{center}
% \includegraphics[width=0.6\textwidth,keepaspectratio]{{VND/n80w20.004/n80w20.004-CpuTime}.pdf}
% \captionof{figure}{n80w20.004 - Runtime boxplots for the different variable neighborhood descent algorithms}
% \end{center}

% \begin{center}
% \includegraphics[width=0.6\textwidth,keepaspectratio]{{VND/n80w20.004/n80w20.004-PRPD}.pdf}
% \captionof{figure}{n80w20.004 - PRPD boxplots for the different variable neighborhood descent algorithms}
% \end{center}

% \begin{center}
% \begin{tabular}{|l|l|}
% \hline
% \textbf{Test} & \textbf{P-Value} \\
% \hline
% Tei vs Tie - Standard&3.95591160889952e-18\\
% \hline
% Tei vs Tie - Piped&3.95591160889952e-18\\
% \hline
% Standard vs Piped - Tei&3.95591160889952e-18\\
% \hline
% Standard vs Piped - Tie&3.95591160889952e-18\\
% \hline
% \end{tabular}
% \captionof{table}{n80w20.004 - Results of Wilcoxon paired signed rank test}
% \label{tab:w.14}
% \end{center}

% \subsubsection{n80w20.005}
% \begin{center}
% \includegraphics[width=0.6\textwidth,keepaspectratio]{{VND/n80w20.005/n80w20.005-CpuTime}.pdf}
% \captionof{figure}{n80w20.005 - Runtime boxplots for the different variable neighborhood descent algorithms}
% \end{center}

% \begin{center}
% \includegraphics[width=0.6\textwidth,keepaspectratio]{{VND/n80w20.005/n80w20.005-PRPD}.pdf}
% \captionof{figure}{n80w20.001 - PRPD boxplots for the different variable neighborhood descent algorithms}
% \end{center}

% \begin{center}
% \begin{tabular}{|l|l|}
% \hline
% \textbf{Test} & \textbf{P-Value} \\
% \hline
% Tei vs Tie - Standard&3.95591160889952e-18\\
% \hline
% Tei vs Tie - Piped&3.95591160889952e-18\\
% \hline
% Standard vs Piped - Tei&3.95591160889952e-18\\
% \hline
% Standard vs Piped - Tie&3.95591160889952e-18\\
% \hline
% \end{tabular}
% \captionof{table}{n80w20.005 - Results of Wilcoxon paired signed rank test}
% \label{tab:w.15}
% \end{center}

% \subsubsection{n80w200.001}
% \begin{center}
% \includegraphics[width=0.6\textwidth,keepaspectratio]{{VND/n80w200.001/n80w200.001-CpuTime}.pdf}
% \captionof{figure}{n80w200.001 - Runtime boxplots for the different variable neighborhood descent algorithms}
% \end{center}

% \begin{center}
% \includegraphics[width=0.6\textwidth,keepaspectratio]{{VND/n80w200.001/n80w200.001-PRPD}.pdf}
% \captionof{figure}{n80w200.001 - PRPD boxplots for the different variable neighborhood descent algorithms}
% \end{center}

% \begin{center}
% \begin{tabular}{|l|l|}
% \hline
% \textbf{Test} & \textbf{P-Value} \\
% \hline
% Tei vs Tie - Standard&4.07730530936212e-18\\
% \hline
% Tei vs Tie - Piped&2.92094064174088e-17\\
% \hline
% Standard vs Piped - Tei&2.72456795287507e-16\\
% \hline
% Standard vs Piped - Tie&3.95591160889952e-18\\
% \hline
% \end{tabular}
% \captionof{table}{n80w200.001 - Results of Wilcoxon paired signed rank test}
% \label{tab:w.16}
% \end{center}

% \subsubsection{n80w200.002}
% \begin{center}
% \includegraphics[width=0.6\textwidth,keepaspectratio]{{VND/n80w200.002/n80w200.002-CpuTime}.pdf}
% \captionof{figure}{n80w200.002 - Runtime boxplots for the different variable neighborhood descent algorithms}
% \end{center}

% \begin{center}
% \includegraphics[width=0.6\textwidth,keepaspectratio]{{VND/n80w200.002/n80w200.002-PRPD}.pdf}
% \captionof{figure}{n80w200.002 - PRPD boxplots for the different variable neighborhood descent algorithms}
% \end{center}

% \begin{center}
% \begin{tabular}{|l|l|}
% \hline
% \textbf{Test} & \textbf{P-Value} \\
% \hline
% Tei vs Tie - Standard&3.95591160889952e-18\\
% \hline
% Tei vs Tie - Piped&1.52379449675399e-17\\
% \hline
% Standard vs Piped - Tei&1.74838327736385e-15\\
% \hline
% Standard vs Piped - Tie&3.95591160889952e-18\\
% \hline
% \end{tabular}
% \captionof{table}{n80w200.002 - Results of Wilcoxon paired signed rank test}
% \label{tab:w.17}
% \end{center}

% \subsubsection{n80w200.003}
% \begin{center}
% \includegraphics[width=0.6\textwidth,keepaspectratio]{{VND/n80w200.003/n80w200.003-CpuTime}.pdf}
% \captionof{figure}{n80w200.003 - Runtime boxplots for the different variable neighborhood descent algorithms}
% \end{center}

% \begin{center}
% \includegraphics[width=0.6\textwidth,keepaspectratio]{{VND/n80w200.003/n80w200.003-PRPD}.pdf}
% \captionof{figure}{n80w200.003 - PRPD boxplots for the different variable neighborhood descent algorithms}
% \end{center}

% \begin{center}
% \begin{tabular}{|l|l|}
% \hline
% \textbf{Test} & \textbf{P-Value} \\
% \hline
% Tei vs Tie - Standard&2.04955667109233e-17\\
% \hline
% Tei vs Tie - Piped&2.59611565456869e-17\\
% \hline
% Standard vs Piped - Tei&1.50422804122146e-07\\
% \hline
% Standard vs Piped - Tie&3.95591160889952e-18\\
% \hline
% \end{tabular}
% \captionof{table}{n80w200.003 - Results of Wilcoxon paired signed rank test}
% \label{tab:w.18}
% \end{center}

% \subsubsection{n80w200.004}
% \begin{center}
% \includegraphics[width=0.6\textwidth,keepaspectratio]{{VND/n80w200.004/n80w200.004-CpuTime}.pdf}
% \captionof{figure}{n80w200.004 - Runtime boxplots for the different variable neighborhood descent algorithms}
% \end{center}

% \begin{center}
% \includegraphics[width=0.6\textwidth,keepaspectratio]{{VND/n80w200.004/n80w200.004-PRPD}.pdf}
% \captionof{figure}{n80w200.004 - PRPD boxplots for the different variable neighborhood descent algorithms}
% \end{center}

% \begin{center}
% \begin{tabular}{|l|l|}
% \hline
% \textbf{Test} & \textbf{P-Value} \\
% \hline
% Tei vs Tie - Standard&4.07730530936212e-18\\
% \hline
% Tei vs Tie - Piped&4.29577057320019e-16\\
% \hline
% Standard vs Piped - Tei&5.3075517052254e-11\\
% \hline
% Standard vs Piped - Tie&3.95591160889952e-18\\
% \hline
% \end{tabular}
% \captionof{table}{n80w200.004 - Results of Wilcoxon paired signed rank test}
% \label{tab:w.19}
% \end{center}

% \subsubsection{n80w200.005}
% \begin{center}
% \includegraphics[width=0.6\textwidth,keepaspectratio]{{VND/n80w200.005/n80w200.005-CpuTime}.pdf}
% \captionof{figure}{n80w200.005 - Runtime boxplots for the different variable neighborhood descent algorithms}
% \end{center}

% \begin{center}
% \includegraphics[width=0.6\textwidth,keepaspectratio]{{VND/n80w200.005/n80w200.005-PRPD}.pdf}
% \captionof{figure}{n80w200.001 - PRPD boxplots for the different variable neighborhood descent algorithms}
% \end{center}

% \begin{center}
% \begin{tabular}{|l|l|}
% \hline
% \textbf{Test} & \textbf{P-Value} \\
% \hline
% Tei vs Tie - Standard&1.39380002081336e-17\\
% \hline
% Tei vs Tie - Piped&4.07730530936212e-18\\
% \hline
% Standard vs Piped - Tei&3.72316935219101e-06\\
% \hline
% Standard vs Piped - Tie&3.95591160889952e-18\\
% \hline
% \end{tabular}
% \captionof{table}{n80w200.001 - Results of Wilcoxon paired signed rank test}
% \label{tab:w.20}
% \end{center}

% \subsection{Statistics}
% \subsubsection{Standard-Transpose-Exchange-Insert}
% \begin{center}
% \begin{tabular}{|l|c|l|l|}
% \hline
% \textbf{Instance}& \textbf{\% Infeasible} & $\mathbf{\bar{PRDP}}$ &$\mathbf{\bar{Runtime}}$\\
% \hline
% n80w20.001&0.71&14772.04644164&50.611339\\
% \hline
% n80w20.002&0.88&12888.542&50.727053\\
% \hline
% n80w20.003&0.92&19936.872&50.820348\\
% \hline
% n80w20.004&0.62&17234.94260984&50.049484\\
% \hline
% n80w20.005&0.94&12564.0560428&50.269182\\
% \hline
% n80w200.001&0.28&11212.97389136&49.151249\\
% \hline
% n80w200.002&0.03&629.5853274&51.433949\\
% \hline
% n80w200.003&0.07&1511.56628539&49.082085\\
% \hline
% n80w200.004&0.16&4193.4817209&49.662512\\
% \hline
% n80w200.005&0.01&466.6729061&46.701953\\
% \hline
% \end{tabular}
% \captionof{table}{Statistics summary for variable neighborhood descent algorithm with Transpose-Exchange-Insert neighborhood chain and Standard VND type}
% \label{tab:s.tei}
% \end{center}

% \subsubsection{Standard-Transpose-Insert-Exchange}
% \begin{center}
% \begin{tabular}{|l|c|l|l|}
% \hline
% \textbf{Instance}& \textbf{\% Infeasible} & $\mathbf{\bar{PRDP}}$ &$\mathbf{\bar{Runtime}}$\\
% \hline
% n80w20.001&0.54&10874.77632472&15.268454\\
% \hline
% n80w20.002&0.62&8411.724&15.386641\\
% \hline
% n80w20.003&0.44&7645.295&15.638153\\
% \hline
% n80w20.004&0.39&7153.68881324&15.980347\\
% \hline
% n80w20.005&0.25&3475.2731712&15.55767\\
% \hline
% n80w200.001&0.16&4898.3227617&33.424555\\
% \hline
% n80w200.002&0&11.0430351&32.198479\\
% \hline
% n80w200.003&0.05&1082.1460308&34.345522\\
% \hline
% n80w200.004&0.28&7804.19186258&32.583152\\
% \hline
% n80w200.005&0&10.20227353&34.501294\\
% \hline
% \end{tabular}
% \captionof{table}{Statistics summary for variable neighborhood descent algorithm with Transpose-Insert-Exchange neighborhood chain and Standard VND type}
% \label{tab:s.tie}
% \end{center}

% \subsubsection{Piped-Transpose-Exchange-Insert}
% \begin{center}
% \begin{tabular}{|l|c|l|l|}
% \hline
% \textbf{Instance}& \textbf{\% Infeasible} & $\mathbf{\bar{PRDP}}$ &$\mathbf{\bar{Runtime}}$\\
% \hline
% n80w20.001&0.59&12336.84228578&35.694416\\
% \hline
% n80w20.002&0.94&15603.0142035&36.212393\\
% \hline
% n80w20.003&0.83&19338.924&34.821217\\
% \hline
% n80w20.004&0.55&13170.33921962&36.438959\\
% \hline
% n80w20.005&0.45&6683.336214&36.202891\\
% \hline
% n80w200.001&0.19&5104.3621015&40.772642\\
% \hline
% n80w200.002&0.01&218.8179842&44.241593\\
% \hline
% n80w200.003&0.06&2584.90674231&44.725066\\
% \hline
% n80w200.004&0.17&3430.64506042&43.760992\\
% \hline
% n80w200.005&0.02&693.0136326&42.646023\\
% \hline
% \end{tabular}
% \captionof{table}{Statistics summary for variable neighborhood descent algorithm with Transpose-Exchange-Insert neighborhood chain and Piped VND type}
% \label{tab:p.tei}
% \end{center}

% \subsubsection{Piped-Transpose-Insert-Exchange}
% \begin{center}
% \begin{tabular}{|l|c|l|l|}
% \hline
% \textbf{Instance}& \textbf{\% Infeasible} & $\mathbf{\bar{PRDP}}$ &$\mathbf{\bar{Runtime}}$\\
% \hline
% n80w20.001&0.68&16393.81210394&24.788225\\
% \hline
% n80w20.002&0.81&11667.654&25.902581\\
% \hline
% n80w20.003&0.84&20537.669&26.442309\\
% \hline
% n80w20.004&0.46&8779.79314651&26.424231\\
% \hline
% n80w20.005&0.24&3876.3661498&26.511156\\
% \hline
% n80w200.001&0.21&5917.0312803&52.302366\\
% \hline
% n80w200.002&0.01&626.0983757&56.238843\\
% \hline
% n80w200.003&0.04&867.92563269&58.498874\\
% \hline
% n80w200.004&0.28&6281.58944822&55.867038\\
% \hline
% n80w200.005&0.01&236.8657243&58.331595\\
% \hline
% \end{tabular}
% \captionof{table}{Statistics summary for variable neighborhood descent algorithm with Transpose-Insert-Exchange neighborhood chain and Piped VND type}
% \label{tab:p.tie}
% \end{center}

% \subsection{Results discussion}
% By looking at tables \ref{tab:s.tei}, \ref{tab:s.tie}, \ref{tab:p.tei}, \ref{tab:p.tie} one can see that:
% \begin{itemize}

% \item For some instances (e.g. $n80w20.002$,$n80w20.003$) the algorithm are not able to converge to a feasible solution, as shown in the corresponding boxplots, since the PRPD distribution is centered around 12000-15000, thus indicating the presence of at least 1 constraint violations in most of the cases.

% \item For some other instances (e.g. $n80w20.004$,$n80w20.005$) the algorithms are able to converge to feasible solutions and to the best-known one, but having a right-skewed distribution towards higher values of PRPD.

% \item For the remaining instances, except for some outlier values, the algorithms are able to converge to the best-known solution in most of the runs , even though the average PRPD is not closer to 0. This is due to the fact that the mean of a distribution is sensible to outliers and the penalisation for a constraint violations is extremely high when compared to the mean value.
      
% \item The algorithm ordering in terms of runtimes is $s.tie < p.tie < p.tei < s.tei$ for the  $n80w20.X$ instances while $s.tie < p.tei < s.tei < p.tie$ for $n80w200.X$ ones. The choice to explore the Insert Neighborhood before the Exchange one allows to reduce the computation time for the $n80w20.X$ instances, with a similar solution quality.

% \item The algorithms are more effective on the $n80w200.X$ instances then the $n80w20.X$ once, since they have a lower percentage of infeasible runs and a lower PRPD.

% \item The standard variable neighborhood descent with Transpose-Insert-Exchange neighborhood chain (s.tie) outperforms all the other algorithms in terms of solution quality and runtime.

% \item Tables \ref{tab:w.11}, \ref{tab:w.12}, \ref{tab:w.13}, \ref{tab:w.14}, \ref{tab:w.15}, \ref{tab:w.16}, \ref{tab:w.17}, \ref{tab:w.18}, \ref{tab:w.19}, \ref{tab:w.20} contain, in any case, p-values considerably smaller than the significance level ($\alpha=0.05$). 

% This implies that the null hypothesis corresponding to the equality of the median values of the differences of the two distributions can be rejected, hence assessing the existence of a statistically significant difference among the solution quality generated by analyzed algorithms.

% \item By looking at the Cpu time, one can see that the instances \emph{n80w20.X} have generally lower runtimes than the \emph{n80w200.X} ones. They can then be considered, with respect to the variable neighborhood descent algorithms, simpler (quickier to solve) instances with respect to the latter.

% \end{itemize}
