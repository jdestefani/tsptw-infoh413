%----------------------------------------------------------------------------------------
%	PROBLEM 1
%----------------------------------------------------------------------------------------

% To have just one problem per page, simply put a \clearpage after each problem
\newpage
\begin{homeworkProblem}
\section{Ant Colony Optimization}
\subsection{Problem statement}
Implement two stochastic local search (SLS) algorithms for the traveling salesman problem with time windows (TPSTW), building on top of the perturbative local search methods from the first implementation exercise.
\begin{enumerate}
  \item Run each algorithm 25 times with different random seed on each instance. Instances will be available from http://iridia.ulb.ac.be/˜stuetzle/Teaching/HO/. As termination criterion, for each instance, use the maximum computation time it takes to run a full VND (implemented in the previous exercise) on the same instance and then multiply this time by 1000 (to allow for long enough runs of the SLS algorithms).
 \item Compute the following statistics for each of the two SLS algorithms and each instance:
 \begin{itemize}
   \item Percentage of runs with constraint violations
   \item Mean penalised relative percentage deviation
 \end{itemize}

\item Produce boxplots of penalised relative percentage deviation.
\item Determine, using statistical tests (in this case, the Wilcoxon test), whether there is a statistically significant difference between the quality of the solutions generated by the two algorithms.
\item Measure, for each of the implemented algorithms on 5 instances, the run-time distributions to reach sufficiently high quality solutions (e.g. best-known solutions available at http://iridia.ulb.ac.be/˜manuel/tsptw-instances\#instances).
Measure the run-time distributions across 25 repetitions using a cut-off time of 10 times the termination criterion above.
\item Produce a written report on the implementation exercise:
\begin{itemize}
  \item Please make sure that each implemented SLS algorithm is appropriately described and that the computational results are carefully interpreted. Justify also the choice of the parameter settings and the choice
of the iterative improvement algorithm for the hybrid SLS algorithm.
  \item Present the results as in the previous implementation exercise (tables, boxplots, statistical tests).
  \item Present graphically the results of the analysis of the run-time distributions.
  \item Interpret appropriately the results and make conclusions on the relative performance of the algorithms across all the benchmark instances studied.
\end{itemize}
\end{enumerate}

\subsection{Introduction} \label{sec:intro}
Ant Colony Optimization is an example of population-based metaheuristic (i.e a set of algorithmic concepts that can be used to define heuristic methods) inspired by the behavior of the ant species \emph{Iridomyrmex humilis}.

To be more precise, these insects are able, by means of stigmergic communication, to choose the shortest path between their nest and a food source, when given the choice (\cite{deneubourg1990self}).

The communication process occurs by deposing a certain quantity of pheromone in the environment that can be sensed by the other ants and that will be used by them as an heuristic (i.e an information to guide their choice) for selecting the shortest path.

Furthermore, the pheromone quantity on a certain location decreases over time because of evaporation, thus requiring a continuous deposit process to be effective.

The convergence to one of the paths will occur as a consequence of the self-reinforcing pheromone deposit mechanism.
In fact the more pheromone is deposited on a path, the more ants will follow the pheromone trail on that path deposing even more pheromone.

The first application of Ant Colony Optimization method, the Ant System, has been made on the optimization version of the Travelling Salesman Problem (TSP) (\cite{dorigo1996ant}).

In this implemetation a population of virtual agents (an ant colony) is used to explore the search space (the virtual environment).

In the same fashion as the real insects, the ants are able to deposit virtual pheromone in the environment, to signal to the other ants the presence of promising solutions.

The general outline of the implemented algorithm is the following: 

\begin{algorithm}[!h]
  \caption{Ant Colony Optimization - Outline}\label{aco}
  \begin{algorithmic}[1]
    \State \emph{InitalizePheromoneTrail} 
    \While{!(TerminationCondition)}
        \State \emph{ConstructAntsSolutions}
        \State \emph{LocalSearch} (Optional)
        \State \emph{UpdatePheromoneTrails}
    \EndWhile
\end{algorithmic}
\end{algorithm}

The design of the solution construction and pheromone update mechanism is the main point of the algorithm.
Implementation details of the basic ACO system, the Ant System can be found in \cite{dorigo2006artificial}.

\subsection{Algorithm structure} \label{sec:algstrucACO}
The proposed algorithm is an implementation of one of the extensions to the Ant Colony Optimization metaheuristic framework, the \maxmin Ant System (cf. \cite{stutzle2000max}).
The main differences with respect to the basic ACO approach are the following:
\begin{itemize}
  \item Only iteration best or best-so-far ants update pheromone.
  \item A local search after the solution generation is used to further improve the solutions found by the ants at each iteration.
  \item $\forall t \text{ } \tau_{\min} < \tau_{i,j}(t) < \tau_{\max}  $ - Pheromone trails have explicit upper and lower limits
  \item Pheromone trails are re-initialized when stagnated.
\end{itemize}

The aforementioned design choice were made because:

\begin{itemize}
    \item The initialzation of the pheromone trails to their upper bound favors diversification at the beginning of each trial.
    \item The pheromone update rule favors exploitation of (intensification on) the best solutions at each iteration of the algorithm.
    \item By bounding the intensity of the pheromone trails, the probability of stagnation (i.e. all the ants converging and exploiting a single sub-optimal tour) is reduced.
    \item If the pheromone trails values for the solution components of a certain tour $s$ are equal to $\tau_{\max}$, the algorithm is said to be converged.
  \end{itemize}
  

\begin{algorithm}[!h]
  \caption{\maxmin Ant System for TSPTW - Outline}\label{maxmintsptw}
  \begin{algorithmic}[1]
    \Procedure{ACO}{$\alpha,\beta,\rho,\tau_0,p_b,t_{\max},f_{best}$}\Comment{The main procedure}
    \Require $N$ - Node set
    \Require $E$ - Edge set 
    \Require $c$ - Edge cost function
    \Require $t$ - Time window function
    \State {\emph{InitalizePheromoneTrail}($\tau_0,n_{cities}$)}
    \While{\emph{!TerminationCondition}($s,t_{max},f_{best}$)}
      \ForAll {Ant $k$}
        \State $s' \gets$ \emph{ConstructSolution}($\alpha,\beta$)
        \If {\emph{IsImproved}($s,s'$)}
          \State $s \gets s' $
        \EndIf 
      \EndFor
      \State $s \gets$\emph{IterativeImprovementIBI}()
      \State \emph{UpdatePheromoneTrails}($\rho,\varepsilon$)
    \EndWhile
    \State \textbf{return} $s$
    \State
  \EndProcedure
\end{algorithmic}
\end{algorithm}

\begin{center}
  
\begin{minipage}{.45\textwidth}
\centering
\begin{tikzpicture}[->,>=stealth',shorten >=1pt,auto,node distance=3cm,
                    semithick]
  \tikzstyle{every state}=[fill=blue!20,draw=none,thick]

  \node[initial,state]    (Init)                   {};
  \node[state]            (CI) [right of=Init]     {\emph{CI}};
  \node[state]            (CS) [above right of=CI] {\emph{CS}};
  \node[state]            (LS) [below right of=CI] {\emph{LS}};
  \node[state,accepting]  (End)[below left of=CI]  {$\Omega$};

  \path (Init) edge              node {DET:\emph{IPT}()} (CI)
        (CI) edge              node {CDET(not \emph{TC})} (CS)
             edge [left]       node {CDET(\emph{TC})} (End)
        (CS) edge [loop above] node {CDET(not \emph{CC})} (CS)
            edge              node {CDET(\emph{CC})} (LS)
        (LS) edge [loop below] node {CDET(not \emph{LO})} (LS)
            edge               node {CDET(\emph{LO})} (CI);
\end{tikzpicture}
\end{minipage}%
\hspace{1.5cm}
\begin{minipage}{.45\textwidth}
\centering
\paragraph{Nodes}
\begin{itemize}
  \item \emph{CI} $\equiv$ Dummy node
  \item \emph{CS} $\equiv$ \emph{ConstructSolution}($\alpha,\beta$)
  \item \emph{LS} $\equiv$ \emph{IterativeImprovementIBI}()
\end{itemize}
\paragraph{Conditions}
\begin{itemize}
  \item \emph{IPT} $\equiv$ \emph{InitializePheromoneTrail}()
  \item \emph{CC} $\equiv$ \emph{ConstructionComplete}()
  \item \emph{LO} $\equiv$ \emph{LocalOptimum}()
  \item \emph{TC} $\equiv$ \emph{TerminationCondition}($s,t_{max},f_{best}$)
\end{itemize}
\end{minipage}
\captionof{figure}{\maxmin  Ant System GLSM}
\end{center}

\newpage
In the implementation, an instance is completely defined by:
\begin{itemize}
\item \textbf{Cost matrix} - Encapsulating information on the node set $N$, edge set $E$, and weighting of each edge $c$.
\item \textbf{Time window vector} - Describing the time window mapping function $t$ for each node.
\end{itemize}

\subsubsection{Pheromone Initialization}
\begin{algorithm}[!h]
  \caption{Pheromone Initialization}\label{init}
  \begin{algorithmic}[1]
    \Procedure{InitalizePheromoneTrail}{$\tau_0,n_{cities}$}
      \State $\tau_{\max} \gets $
      \State $\tau_{\min} \gets $
      \State $i \gets 0$
      \State $j \gets 0$
      \For{$i < n_{cities}$} 
        \For{$j < i$} 
          \State $\tau_{ij} \gets \tau_0$
          \State $\tau_{ji} \gets \tau_{ij}$
          \State $ j \gets j + 1$  
        \EndFor
        \State $ i \gets i + 1$ 
      \EndFor
    \EndProcedure
\end{algorithmic}
\end{algorithm}

As discussed in \nameref{sec:intro}, the ACO methods are based on stigmergic communication among the agents by means of virtual pheromone.

While the real ants can deposit pheromone anywhere in the environment, the virtual ants may only exchange informations concerning solutions components.

For this reason, every admissible edge $e_{i,j}$ of $E$ has an associated pheromone value $\tau_{i,j}$, that have to be initialized at the beginning of the execution of the algorithm.

The inizialization value $\tau_0$ is a parameter of the algorithm, for the \maxmin Ant System $\tau_0 = \tau_{i,j}(0) = \tau_{\max}$. 


\subsubsection{Solution construction}
\begin{algorithm}[!h]
  \caption{Solution Construction}\label{sol}
  \begin{algorithmic}[1]
    \Procedure{ConstructSolution}{$\alpha,\beta$}\Comment{The main procedure}
      \State \Comment{$s_i$ reprents the $i^{th}$ component of the solution}
      \State $s_0 \gets 0$ \Comment{Every solution starts at the depot}
      \State $s_1 \gets$ \emph{RandomCitySelection}() \Comment{Random choice of the starting city}
      \State $i \gets 1$
      \While{!\emph{SolutionComplete}()}
        \State $s_i \gets $ \emph{RouletteWheelSelection}() \Comment{ $s_i$ stochastically chosen according to the probability distribution defined by \ref{eq:tranprob}} 
        \State $i \gets i+1$
      \EndWhile
    \EndProcedure
\end{algorithmic}
\end{algorithm}

The solution construction process, used by every ant $k$ in the system, consist of a probabilistic selection of solution components.
Every edge $e_{i,j}$ has a selection probability $p_{i,j}^k(t)$ (also called transition probability) defined as follows:

\begin{equation} \label{eq:tranprob}
p_{i,j}^k(t) = \begin{cases}
  \frac{[\tau_{i,j}(t)]^\alpha \cdot [\eta_{i,j}]^\beta}{\sum_{k} \in A(s_{i}) [\tau_{k,j}(t)]^\alpha \cdot [\eta_{k,j}]^\beta} & j \in A(s_{i}) \\
 0 & \text{otherwise} \\
\end{cases}
\end{equation}

As one can see in \ref{eq:tranprob}, the transition probability is determined by a constant, locally available heuristic information $\eta_{i,j}$ and by the time varying pheromone trail $\tau_{i,j}(t)$.
This probability is defined on the set $A(s_i)$ of available (i.e. not yet visited) cities while visiting solution component $s_i$.
The value of the parameters $\alpha$ and $\beta$ determines the relative imporance of the heuristic information and the pheromone trail, respectively.

\subsubsection{Solution improvement}
\begin{algorithm}[!h]
  \caption{Solution improvement}\label{sol}
  \begin{algorithmic}[1]
    \Procedure{IsImproving}{$s,s'$}\Comment{The main procedure}
      \If {$\Omega(s') < \Omega(s)$}
          \State \textbf{return true}
      \Else
           \If {$\Omega(s') = \Omega(s) \wedge f(s') < f(s)$}
            \State \textbf{return true}
           \EndIf
      \EndIf
      \State \textbf{return false}
      \EndProcedure
\end{algorithmic}
\end{algorithm}

A solution $s'$ is considered improving the current best solution $s$ if and only if:
\begin{itemize}
  \item Either,it has a smaller number of constraints violation (i.e. $\Omega(s') < \Omega(s)$)
  \item Or, it has the same number of constraints violations ($\Omega(s') == \Omega(s)$) but the total tour duration is smaller ($f(s') < f(s)$).
\end{itemize}


\subsubsection{Admissible heuristics}
The heuristic component $\eta_{i,j}$ is used to guide the selection of solution compoments towards those components that are included in optimal solution.

\paragraph{Dorigo et al, 1996}
The heuristic originally proposed by Dorigo et al, in \cite{dorigo1996ant}, for the TSP problem is:
\begin{equation}
  \eta_{i,j} = \frac{1}{c(e_{i,j})}
\end{equation}

The main idea behind this heuristic is that, if the selection process tends to select, at each step, the shortest connection between the current node and the following, the built tour should be of the shortest length.
This heuristic is cited for explanation purposes, even though it cannot be used for the TSPTW problem, since it will guide the exploration only towards shorter solutions, without taking into account the presence of the time windows.

\paragraph{Cheng and Mao, 2007}
The local heuristics used in \cite{cheng2007modified} are similar to that proposed by Gambardella et al. \cite{gambardella1999macs} in their multiple ant colony system (MACS) designed to solve the vehicle routing problem with time windows (VRPTW).

\begin{equation}
[\eta_{i,j}]^\beta = [g_{i,j}]^\beta \cdot [h_{i,j}]^\gamma
\end{equation}  

The two components $g_{i,j}$ and $h_{i,j}$, are designed, respectively, to avoid lateness (that is, arriving in the node where the time windows is already terminated) and waiting times (i.e. arriving in the node before the time windows open).

\subparagraph{Lateness avoidance}
\begin{equation}
g_{i,j} = \begin{cases}
 \frac{1}{1+e^{\delta \cdot (G_{i,j} - \mu)}}  &  G_{i,j} = b_j - t_j \geq 0 \\
0 & \text{otherwise} \\
\end{cases}
\end{equation}
where
\begin{itemize}
  \item $G_{i,j} = b_j - t_j$ - Slack corresponding to the time window $j$ while being in node $i$
  \item $t_i$ - Arrival time at node $i$
  \item $b_i$ - Closing time of time window $i$
  \item $G(i) = \{k\text{ } | \text{ }G_{i,k} \geq 0\}$ - Set of feasible neighbors of node $i$ (i.e. such that node $k$ is reached earlier than its closing time)
  \item $\mu = \frac{1}{|G(i)|}\sum_{j \in G(i)}  G_{i,j}$ - Average slack 
  \item $\delta$ - Parameter to control the slope of the sigmoidal function
\end{itemize}

\subparagraph{Waiting time avoidance}
\begin{equation}
h_{i,j} = \begin{cases}
 \frac{1}{1+e^{\lambda \cdot (H_{i,j} - \upsilon)}}  &  H_{i,j} = t_j - a_j \geq 0 \\
0 & \text{otherwise} \\
\end{cases}
\end{equation}
where
\begin{itemize}
  \item $H_{i,j} = t_j - a_j$ - Waiting time corresponding to the time window $j$ while being in node $i$
  \item $t_i$ - Arrival time at node $i$
  \item $a_i$ - Opening time of time window $i$
  \item $H(i) = \{k\text{ } | \text{ }H_{i,k} \geq 0\}$ - Set of non-waiting neighbors of node $i$ (i.e such that node $k$ is reached within the time window)
  \item $\upsilon = \frac{1}{|H(i)|}\sum_{j \in H(i)}  H_{i,j}$ - Average waiting time 
  $\lambda$ - Parameter to control the slope of the sigmoidal function 
\end{itemize}


\paragraph{Lopez-Ibanez and Blum, 2010}
The approach used in \cite{lopez2010beam} is a linear combination based on $\lambda_i$ coefficients, of the normalized values of opening and closing times of the time windows and the travelling cost from one city to another.

\begin{equation} \label{eq:heuristic}
\eta_{i-1,k} = \lambda_{a} \cdot \frac{a_{\max}-a_{k}}{a_{\max}-a_{\min}} + \lambda_{b} \cdot \frac{b_{\max}-b_{k}}{b_{\max}-b_{\min}} + \lambda_{c} \cdot \frac{c_{\max}-c_{i-1,k}}{c_{\max}-c_{\min}}
\end{equation}

where
\begin{itemize}
  \item $a_i$ - Opening time of time window $i$
  \item $a_{\max} = \max_{j \in N} a_{j}$ - Maximum time window opening time in the neighborhood of node $i$
  \item $a_{\min} = \min_{j \in N} a_{j}$ - Minimum time window opening time in the neighborhood of node $i$
  \item $b_i$ - Closing time of time window $i$
  \item $b_{\max} = \max_{j \in N} b_{j}$ - Maximum time window closing time in the neighborhood of node $i$
  \item $b_{\min} = \min_{j \in N} b_{j}$ - Minimum time window closing time in the neighborhood of node $i$
  \item $c_{i,j}$ - Travelling cost from node $i$ to node $j$
  \item $c_{\max} = \max_j c_{i,j}$ - Maximum travelling cost from node $i$ 
  \item $c_{\min} = \min_j c_{i,j}$ - Minimum travelling cost from node $i$
  \item $\lambda_{a},\lambda_{b},\lambda_{c}$ s.t. $\lambda_{a}+\lambda_{b}+\lambda_{c}=1$ - Randomly selected weights
\end{itemize}

In this implementation, the heuristic information will be computed according to \cite{lopez2010beam}

\subsubsection{Pheromone trails update}
\begin{algorithm}[!h]
  \caption{Pheromone Trails Update}\label{update}
  \begin{algorithmic}[1]
    \Procedure{UpdatePheromoneTrails}{$\rho,p_b$}\Comment{The main procedure}
     \State $i \gets 0$
      \State $j \gets 0$
      \For{$i < n_{cities}$} 
        \For{$j < i$}
          \If{\emph{Random}() $< \varepsilon$} 
          \State $\tau_{ij} \gets (1-\rho)\cdot\tau_{ij}+\Delta\tau_{i,j}^{Bi}$
          \Else
          \State $\tau_{ij} \gets (1-\rho)\cdot\tau_{ij}+\Delta\tau_{i,j}^{Bo}$
          \EndIf
          \If{$\tau_{ij} < \tau_{\min}$} 
            \State $\tau_{ij} \gets \tau_{\min}$
          \EndIf
          \If{$\tau_{ij} > \tau_{\max}$} 
            \State $\tau_{ij} \gets \tau_{\max}$
          \EndIf
          \State $\tau_{ji} \gets \tau_{ij}$
          \State $ j \gets j + 1$  
        \EndFor
        \State $ i \gets i + 1$ 
      \EndFor
    \EndProcedure
\end{algorithmic}
\end{algorithm}

As discussed in \nameref{sec:algstrucACO}, the pheromone update will be made by a single ant, being either the one who found the best solution in the current iteration: 

\begin{equation}
  \Delta\tau_{i,j}^{Bi} = \begin{cases}
    \frac{1}{T_{d}^{\text{Bi}}} & e_{i,j} \in T^{\text{Bi}}  \\
    0 & \text{otherwise} 
      \end{cases}
\end{equation}

Or the one having found the global best (best-so-far) solution:

\begin{equation}
  \Delta\tau_{i,j}^{Bo} = \begin{cases}
    \frac{1}{T_{d}^{\text{Bo}}} & e_{i,j} \in T^{\text{Bo}}  \\
    0 & \text{otherwise} 
  \end{cases}
\end{equation}

where
\begin{itemize}
\item $e_{i,j}$ - Edge connecting node $i$ and $j$
\item $T_{d}^{i}$ - Complete tour duration of tour $i$
\item $T^{\text{Bi}}$ - Best tour of the current iteration
\item $T^{\text{Bo}}$ - Best tour overall 
\end{itemize}


Provided that the best solution is known, the bounds on the pheromone values can be estimated by:
\begin{equation}
\begin{array}{lccr}
  \hat{\tau}_{\max} = \frac{1}{\rho \cdot T_{d}^{\text{Go}}} & & & \hat{\tau}_{\min} = \frac{\hat{\tau}_{\max}}{a}
\end{array}
\end{equation}

\subsubsection{Local search}
\begin{algorithm}[!h]
  \caption{Iterative Improvement - (Insert neighborhood with best improve pivoting rule)}\label{locsearch}
  \begin{algorithmic}[1]
    \Procedure{IterativeImprovementIBI}{$s$}\Comment{The main procedure}
      \State $s^{*} \gets s$ 
      \For {$i \in \{1,\cdots,|N|\}$}
        \For {$j \in \{1,\cdots,|N|\}$}
          \If{ $i = j \vee  j = i-1 $}
				    \State \textbf{continue}
			    \EndIf
			    \State $s' \gets$ \emph{InsertTourComponent}($s,i,j$)
			    \If {\emph{IsImproving}($s^{*},s'$)}
          \State $s^{*} \gets s'$
        \EndIf
        \EndFor
      \EndFor
      \State \textbf{return} $s^{*}$
    \EndProcedure
\end{algorithmic}
\end{algorithm}

\begin{algorithm}[!h]
  \caption{Insert neighbor solution computation}\label{locsearch}
  \begin{algorithmic}[1]
    \Procedure{InsertTourComponent}{$s,i,j$}\Comment{The main procedure}
      \State \Comment{$s_i$ reprents the $i^{th}$ component of the solution}
      \State $e \gets s_i$
      \If{i<j}
      \State $k \gets i$
        \While{$k < j$}
				  \State $s_i \gets s_{i+1}$
				  \State $k \gets k+1$
			  \EndWhile
			\Else
			\State $k \gets i$
        \While{$k > j$}
				  \State $s_i \gets s_{i-1}$
				  \State $k \gets k-1$
			  \EndWhile
			\EndIf
			\State $s_j \gets e$
     \State \textbf{return} $s$
    \EndProcedure
\end{algorithmic}
\end{algorithm}


The solution construction process, used by every ant $k$ in the system, consist of a probabilistic selection of solution components.
Every edge $e_{i,j}$ has a selection probability $p_{i,j}^k(t)$ (also called transition probability) defined as follows:

\begin{equation} 
p_{i,j}^k(t) = \begin{cases}
  \frac{[\tau_{i,j}(t)]^\alpha \cdot [\eta_{i,j}]^\beta}{\sum_{k} \in A(s_{i}) [\tau_{k,j}(t)]^\alpha \cdot [\eta_{k,j}]^\beta} & j \in A(s_{i}) \\
 0 & \text{otherwise} \\
\end{cases}
\end{equation}

As one can see in \ref{eq:tranprob}, the transition probability is determined by a constant, locally available heuristic information $\eta_{i,j}$ and by the time varying pheromone trail $\tau_{i,j}(t)$.
This probability is defined on the set $A(s_i)$ of available (i.e. not yet visited) cities while visiting solution component $s_i$.
The value of the parameters $\alpha$ and $\beta$ determines the relative imporance of the heuristic information and the pheromone trail, respectively.

\subsubsection{Termination condition}
\begin{algorithm}[!h]
  \caption{Termination Condition}\label{termcond}
  \begin{algorithmic}[1]
    \Procedure{TerminationCondintion}{$s,t_{\max},s_{best}$}
          \If{ $(f(s) = f_{best} \wedge \Omega(s) = 0) \vee  t > t_{\max} $}
				    \State \textbf{return true}
			    \EndIf
      \State \textbf{return false}
    \EndProcedure
\end{algorithmic}
\end{algorithm}

 
% An Iterative Improvement algorithm generally starts from a candidate solution, which can be either generated randomly or using an heuristic, and improves the evaluation of the solution at each step by modifying the solution structure , until a local optimum is reached.

% In the previous problem, I considered different kinds of 2-opt neighborhood, and different pivoting rules.

% This means that, at each step, a new solution is constructed from the current best by modifying only two solution components (with Transpose,Exchange or Insert operations) and only the first/best improving solution will become the new best soluion.

% The main limitation of such kind of algorithms is that they tend to get stuck in solutions that are locally optimum but not globally.

% Provided that:
% \begin{itemize}
% \item A global optimum is optimal with respect to any kind of neighborhood. 
% \item A solution that is locally optimal with respect to a neighborhood may not be optimal with respect to other kinds of neighborhood
% \end{itemize}
% by dynamically changing the neighborhood type an algorithm is able to escape local optima.

% This section will analyse the results of the execution of two variable neighborhood descent algorithm, based on the previously analyzed iterative improvement algorithms :
% \begin{itemize}
% \item \textbf{Standard Variable Neighborhood Descent} (i.e. Changing neighborhood when a local optimum is encoutered, until the neighborhood chain is terminated and going back to the smallest neighborhood every time the local optimum is escaped.)
% \item \textbf{Piped Variable Neighborhood Descent} (i.e. Using the locally optimum solution found using one neighborhood type in the chain as the initial solution for the following type.)
% \end{itemize}
% The same metrics as in \ref{subsec:metric} will be used to evaluate the algorithms.

% \subsection{Experiment results}
% \subsubsection{n80w20.001}
% \begin{center}
% \includegraphics[width=0.6\textwidth,keepaspectratio]{{VND/n80w20.001/n80w20.001-CpuTime}.pdf}
% \captionof{figure}{n80w20.001 - Runtime boxplots for the different variable neighborhood descent algorithms}
% \end{center}

% \begin{center}
% \includegraphics[width=0.6\textwidth,keepaspectratio]{{VND/n80w20.001/n80w20.001-PRPD}.pdf}
% \captionof{figure}{n80w20.001 - PRPD boxplots for the different variable neighborhood descent algorithms}
% \end{center}

% \begin{center}
% \begin{tabular}{|l|l|}
% \hline
% \textbf{Test} & \textbf{P-Value} \\
% \hline
% Tei vs Tie - Standard&3.95591160889952e-18\\
% \hline
% Tei vs Tie - Piped&3.9556885406462e-18\\
% \hline
% Standard vs Piped - Tei&3.95591160889952e-18\\
% \hline
% Standard vs Piped - Tie&3.95591160889952e-18\\
% \hline
% \end{tabular}
% \captionof{table}{n80w20.001 - Results of Wilcoxon paired signed rank test}
% \label{tab:w.11}
% \end{center}

% \subsubsection{n80w20.002}
% \begin{center}
% \includegraphics[width=0.6\textwidth,keepaspectratio]{{VND/n80w20.002/n80w20.002-CpuTime}.pdf}
% \captionof{figure}{n80w20.002 - Runtime boxplots for the different variable neighborhood descent algorithms}
% \end{center}

% \begin{center}
% \includegraphics[width=0.6\textwidth,keepaspectratio]{{VND/n80w20.002/n80w20.002-PRPD}.pdf}
% \captionof{figure}{n80w20.002 - PRPD boxplots for the different variable neighborhood descent algorithms}
% \end{center}

% \begin{center}
% \begin{tabular}{|l|l|}
% \hline
% \textbf{Test} & \textbf{P-Value} \\
% \hline
% Tei vs Tie - Standard&3.9556885406462e-18\\
% \hline
% Tei vs Tie - Piped&3.95591160889952e-18\\
% \hline
% Standard vs Piped - Tei&3.95591160889952e-18\\
% \hline
% Standard vs Piped - Tie&3.95591160889952e-18\\
% \hline
% \end{tabular}
% \captionof{table}{n80w20.002 - Results of Wilcoxon paired signed rank test}
% \label{tab:w.12}
% \end{center}

% \subsubsection{n80w20.003}
% \begin{center}
% \includegraphics[width=0.6\textwidth,keepaspectratio]{{VND/n80w20.003/n80w20.003-CpuTime}.pdf}
% \captionof{figure}{n80w20.003 - Runtime boxplots for the different variable neighborhood descent algorithms}
% \end{center}

% \begin{center}
% \includegraphics[width=0.6\textwidth,keepaspectratio]{{VND/n80w20.003/n80w20.003-PRPD}.pdf}
% \captionof{figure}{n80w20.003 - PRPD boxplots for the different variable neighborhood descent algorithms}
% \end{center}

% \begin{center}
% \begin{tabular}{|l|l|}
% \hline
% \textbf{Test} & \textbf{P-Value} \\
% \hline
% Tei vs Tie - Standard&3.9552424399092e-18\\
% \hline
% Tei vs Tie - Piped&3.95591160889952e-18\\
% \hline
% Standard vs Piped - Tei&3.95591160889952e-18\\
% \hline
% Standard vs Piped - Tie&3.95591160889952e-18\\
% \hline
% \end{tabular}
% \captionof{table}{n80w20.003 - Results of Wilcoxon paired signed rank test}
% \label{tab:w.13}
% \end{center}

% \subsubsection{n80w20.004}
% \begin{center}
% \includegraphics[width=0.6\textwidth,keepaspectratio]{{VND/n80w20.004/n80w20.004-CpuTime}.pdf}
% \captionof{figure}{n80w20.004 - Runtime boxplots for the different variable neighborhood descent algorithms}
% \end{center}

% \begin{center}
% \includegraphics[width=0.6\textwidth,keepaspectratio]{{VND/n80w20.004/n80w20.004-PRPD}.pdf}
% \captionof{figure}{n80w20.004 - PRPD boxplots for the different variable neighborhood descent algorithms}
% \end{center}

% \begin{center}
% \begin{tabular}{|l|l|}
% \hline
% \textbf{Test} & \textbf{P-Value} \\
% \hline
% Tei vs Tie - Standard&3.95591160889952e-18\\
% \hline
% Tei vs Tie - Piped&3.95591160889952e-18\\
% \hline
% Standard vs Piped - Tei&3.95591160889952e-18\\
% \hline
% Standard vs Piped - Tie&3.95591160889952e-18\\
% \hline
% \end{tabular}
% \captionof{table}{n80w20.004 - Results of Wilcoxon paired signed rank test}
% \label{tab:w.14}
% \end{center}

% \subsubsection{n80w20.005}
% \begin{center}
% \includegraphics[width=0.6\textwidth,keepaspectratio]{{VND/n80w20.005/n80w20.005-CpuTime}.pdf}
% \captionof{figure}{n80w20.005 - Runtime boxplots for the different variable neighborhood descent algorithms}
% \end{center}

% \begin{center}
% \includegraphics[width=0.6\textwidth,keepaspectratio]{{VND/n80w20.005/n80w20.005-PRPD}.pdf}
% \captionof{figure}{n80w20.001 - PRPD boxplots for the different variable neighborhood descent algorithms}
% \end{center}

% \begin{center}
% \begin{tabular}{|l|l|}
% \hline
% \textbf{Test} & \textbf{P-Value} \\
% \hline
% Tei vs Tie - Standard&3.95591160889952e-18\\
% \hline
% Tei vs Tie - Piped&3.95591160889952e-18\\
% \hline
% Standard vs Piped - Tei&3.95591160889952e-18\\
% \hline
% Standard vs Piped - Tie&3.95591160889952e-18\\
% \hline
% \end{tabular}
% \captionof{table}{n80w20.005 - Results of Wilcoxon paired signed rank test}
% \label{tab:w.15}
% \end{center}

% \subsubsection{n80w200.001}
% \begin{center}
% \includegraphics[width=0.6\textwidth,keepaspectratio]{{VND/n80w200.001/n80w200.001-CpuTime}.pdf}
% \captionof{figure}{n80w200.001 - Runtime boxplots for the different variable neighborhood descent algorithms}
% \end{center}

% \begin{center}
% \includegraphics[width=0.6\textwidth,keepaspectratio]{{VND/n80w200.001/n80w200.001-PRPD}.pdf}
% \captionof{figure}{n80w200.001 - PRPD boxplots for the different variable neighborhood descent algorithms}
% \end{center}

% \begin{center}
% \begin{tabular}{|l|l|}
% \hline
% \textbf{Test} & \textbf{P-Value} \\
% \hline
% Tei vs Tie - Standard&4.07730530936212e-18\\
% \hline
% Tei vs Tie - Piped&2.92094064174088e-17\\
% \hline
% Standard vs Piped - Tei&2.72456795287507e-16\\
% \hline
% Standard vs Piped - Tie&3.95591160889952e-18\\
% \hline
% \end{tabular}
% \captionof{table}{n80w200.001 - Results of Wilcoxon paired signed rank test}
% \label{tab:w.16}
% \end{center}

% \subsubsection{n80w200.002}
% \begin{center}
% \includegraphics[width=0.6\textwidth,keepaspectratio]{{VND/n80w200.002/n80w200.002-CpuTime}.pdf}
% \captionof{figure}{n80w200.002 - Runtime boxplots for the different variable neighborhood descent algorithms}
% \end{center}

% \begin{center}
% \includegraphics[width=0.6\textwidth,keepaspectratio]{{VND/n80w200.002/n80w200.002-PRPD}.pdf}
% \captionof{figure}{n80w200.002 - PRPD boxplots for the different variable neighborhood descent algorithms}
% \end{center}

% \begin{center}
% \begin{tabular}{|l|l|}
% \hline
% \textbf{Test} & \textbf{P-Value} \\
% \hline
% Tei vs Tie - Standard&3.95591160889952e-18\\
% \hline
% Tei vs Tie - Piped&1.52379449675399e-17\\
% \hline
% Standard vs Piped - Tei&1.74838327736385e-15\\
% \hline
% Standard vs Piped - Tie&3.95591160889952e-18\\
% \hline
% \end{tabular}
% \captionof{table}{n80w200.002 - Results of Wilcoxon paired signed rank test}
% \label{tab:w.17}
% \end{center}

% \subsubsection{n80w200.003}
% \begin{center}
% \includegraphics[width=0.6\textwidth,keepaspectratio]{{VND/n80w200.003/n80w200.003-CpuTime}.pdf}
% \captionof{figure}{n80w200.003 - Runtime boxplots for the different variable neighborhood descent algorithms}
% \end{center}

% \begin{center}
% \includegraphics[width=0.6\textwidth,keepaspectratio]{{VND/n80w200.003/n80w200.003-PRPD}.pdf}
% \captionof{figure}{n80w200.003 - PRPD boxplots for the different variable neighborhood descent algorithms}
% \end{center}

% \begin{center}
% \begin{tabular}{|l|l|}
% \hline
% \textbf{Test} & \textbf{P-Value} \\
% \hline
% Tei vs Tie - Standard&2.04955667109233e-17\\
% \hline
% Tei vs Tie - Piped&2.59611565456869e-17\\
% \hline
% Standard vs Piped - Tei&1.50422804122146e-07\\
% \hline
% Standard vs Piped - Tie&3.95591160889952e-18\\
% \hline
% \end{tabular}
% \captionof{table}{n80w200.003 - Results of Wilcoxon paired signed rank test}
% \label{tab:w.18}
% \end{center}

% \subsubsection{n80w200.004}
% \begin{center}
% \includegraphics[width=0.6\textwidth,keepaspectratio]{{VND/n80w200.004/n80w200.004-CpuTime}.pdf}
% \captionof{figure}{n80w200.004 - Runtime boxplots for the different variable neighborhood descent algorithms}
% \end{center}

% \begin{center}
% \includegraphics[width=0.6\textwidth,keepaspectratio]{{VND/n80w200.004/n80w200.004-PRPD}.pdf}
% \captionof{figure}{n80w200.004 - PRPD boxplots for the different variable neighborhood descent algorithms}
% \end{center}

% \begin{center}
% \begin{tabular}{|l|l|}
% \hline
% \textbf{Test} & \textbf{P-Value} \\
% \hline
% Tei vs Tie - Standard&4.07730530936212e-18\\
% \hline
% Tei vs Tie - Piped&4.29577057320019e-16\\
% \hline
% Standard vs Piped - Tei&5.3075517052254e-11\\
% \hline
% Standard vs Piped - Tie&3.95591160889952e-18\\
% \hline
% \end{tabular}
% \captionof{table}{n80w200.004 - Results of Wilcoxon paired signed rank test}
% \label{tab:w.19}
% \end{center}

% \subsubsection{n80w200.005}
% \begin{center}
% \includegraphics[width=0.6\textwidth,keepaspectratio]{{VND/n80w200.005/n80w200.005-CpuTime}.pdf}
% \captionof{figure}{n80w200.005 - Runtime boxplots for the different variable neighborhood descent algorithms}
% \end{center}

% \begin{center}
% \includegraphics[width=0.6\textwidth,keepaspectratio]{{VND/n80w200.005/n80w200.005-PRPD}.pdf}
% \captionof{figure}{n80w200.001 - PRPD boxplots for the different variable neighborhood descent algorithms}
% \end{center}

% \begin{center}
% \begin{tabular}{|l|l|}
% \hline
% \textbf{Test} & \textbf{P-Value} \\
% \hline
% Tei vs Tie - Standard&1.39380002081336e-17\\
% \hline
% Tei vs Tie - Piped&4.07730530936212e-18\\
% \hline
% Standard vs Piped - Tei&3.72316935219101e-06\\
% \hline
% Standard vs Piped - Tie&3.95591160889952e-18\\
% \hline
% \end{tabular}
% \captionof{table}{n80w200.001 - Results of Wilcoxon paired signed rank test}
% \label{tab:w.20}
% \end{center}

% \subsection{Statistics}
% \subsubsection{Standard-Transpose-Exchange-Insert}
% \begin{center}
% \begin{tabular}{|l|c|l|l|}
% \hline
% \textbf{Instance}& \textbf{\% Infeasible} & $\mathbf{\bar{PRDP}}$ &$\mathbf{\bar{Runtime}}$\\
% \hline
% n80w20.001&0.71&14772.04644164&50.611339\\
% \hline
% n80w20.002&0.88&12888.542&50.727053\\
% \hline
% n80w20.003&0.92&19936.872&50.820348\\
% \hline
% n80w20.004&0.62&17234.94260984&50.049484\\
% \hline
% n80w20.005&0.94&12564.0560428&50.269182\\
% \hline
% n80w200.001&0.28&11212.97389136&49.151249\\
% \hline
% n80w200.002&0.03&629.5853274&51.433949\\
% \hline
% n80w200.003&0.07&1511.56628539&49.082085\\
% \hline
% n80w200.004&0.16&4193.4817209&49.662512\\
% \hline
% n80w200.005&0.01&466.6729061&46.701953\\
% \hline
% \end{tabular}
% \captionof{table}{Statistics summary for variable neighborhood descent algorithm with Transpose-Exchange-Insert neighborhood chain and Standard VND type}
% \label{tab:s.tei}
% \end{center}

% \subsubsection{Standard-Transpose-Insert-Exchange}
% \begin{center}
% \begin{tabular}{|l|c|l|l|}
% \hline
% \textbf{Instance}& \textbf{\% Infeasible} & $\mathbf{\bar{PRDP}}$ &$\mathbf{\bar{Runtime}}$\\
% \hline
% n80w20.001&0.54&10874.77632472&15.268454\\
% \hline
% n80w20.002&0.62&8411.724&15.386641\\
% \hline
% n80w20.003&0.44&7645.295&15.638153\\
% \hline
% n80w20.004&0.39&7153.68881324&15.980347\\
% \hline
% n80w20.005&0.25&3475.2731712&15.55767\\
% \hline
% n80w200.001&0.16&4898.3227617&33.424555\\
% \hline
% n80w200.002&0&11.0430351&32.198479\\
% \hline
% n80w200.003&0.05&1082.1460308&34.345522\\
% \hline
% n80w200.004&0.28&7804.19186258&32.583152\\
% \hline
% n80w200.005&0&10.20227353&34.501294\\
% \hline
% \end{tabular}
% \captionof{table}{Statistics summary for variable neighborhood descent algorithm with Transpose-Insert-Exchange neighborhood chain and Standard VND type}
% \label{tab:s.tie}
% \end{center}

% \subsubsection{Piped-Transpose-Exchange-Insert}
% \begin{center}
% \begin{tabular}{|l|c|l|l|}
% \hline
% \textbf{Instance}& \textbf{\% Infeasible} & $\mathbf{\bar{PRDP}}$ &$\mathbf{\bar{Runtime}}$\\
% \hline
% n80w20.001&0.59&12336.84228578&35.694416\\
% \hline
% n80w20.002&0.94&15603.0142035&36.212393\\
% \hline
% n80w20.003&0.83&19338.924&34.821217\\
% \hline
% n80w20.004&0.55&13170.33921962&36.438959\\
% \hline
% n80w20.005&0.45&6683.336214&36.202891\\
% \hline
% n80w200.001&0.19&5104.3621015&40.772642\\
% \hline
% n80w200.002&0.01&218.8179842&44.241593\\
% \hline
% n80w200.003&0.06&2584.90674231&44.725066\\
% \hline
% n80w200.004&0.17&3430.64506042&43.760992\\
% \hline
% n80w200.005&0.02&693.0136326&42.646023\\
% \hline
% \end{tabular}
% \captionof{table}{Statistics summary for variable neighborhood descent algorithm with Transpose-Exchange-Insert neighborhood chain and Piped VND type}
% \label{tab:p.tei}
% \end{center}

% \subsubsection{Piped-Transpose-Insert-Exchange}
% \begin{center}
% \begin{tabular}{|l|c|l|l|}
% \hline
% \textbf{Instance}& \textbf{\% Infeasible} & $\mathbf{\bar{PRDP}}$ &$\mathbf{\bar{Runtime}}$\\
% \hline
% n80w20.001&0.68&16393.81210394&24.788225\\
% \hline
% n80w20.002&0.81&11667.654&25.902581\\
% \hline
% n80w20.003&0.84&20537.669&26.442309\\
% \hline
% n80w20.004&0.46&8779.79314651&26.424231\\
% \hline
% n80w20.005&0.24&3876.3661498&26.511156\\
% \hline
% n80w200.001&0.21&5917.0312803&52.302366\\
% \hline
% n80w200.002&0.01&626.0983757&56.238843\\
% \hline
% n80w200.003&0.04&867.92563269&58.498874\\
% \hline
% n80w200.004&0.28&6281.58944822&55.867038\\
% \hline
% n80w200.005&0.01&236.8657243&58.331595\\
% \hline
% \end{tabular}
% \captionof{table}{Statistics summary for variable neighborhood descent algorithm with Transpose-Insert-Exchange neighborhood chain and Piped VND type}
% \label{tab:p.tie}
% \end{center}

% \subsection{Results discussion}
% By looking at tables \ref{tab:s.tei}, \ref{tab:s.tie}, \ref{tab:p.tei}, \ref{tab:p.tie} one can see that:
% \begin{itemize}

% \item For some instances (e.g. $n80w20.002$,$n80w20.003$) the algorithm are not able to converge to a feasible solution, as shown in the corresponding boxplots, since the PRPD distribution is centered around 12000-15000, thus indicating the presence of at least 1 constraint violations in most of the cases.

% \item For some other instances (e.g. $n80w20.004$,$n80w20.005$) the algorithms are able to converge to feasible solutions and to the best-known one, but having a right-skewed distribution towards higher values of PRPD.

% \item For the remaining instances, except for some outlier values, the algorithms are able to converge to the best-known solution in most of the runs , even though the average PRPD is not closer to 0. This is due to the fact that the mean of a distribution is sensible to outliers and the penalisation for a constraint violations is extremely high when compared to the mean value.
      
% \item The algorithm ordering in terms of runtimes is $s.tie < p.tie < p.tei < s.tei$ for the  $n80w20.X$ instances while $s.tie < p.tei < s.tei < p.tie$ for $n80w200.X$ ones. The choice to explore the Insert Neighborhood before the Exchange one allows to reduce the computation time for the $n80w20.X$ instances, with a similar solution quality.

% \item The algorithms are more effective on the $n80w200.X$ instances then the $n80w20.X$ once, since they have a lower percentage of infeasible runs and a lower PRPD.

% \item The standard variable neighborhood descent with Transpose-Insert-Exchange neighborhood chain (s.tie) outperforms all the other algorithms in terms of solution quality and runtime.

% \item Tables \ref{tab:w.11}, \ref{tab:w.12}, \ref{tab:w.13}, \ref{tab:w.14}, \ref{tab:w.15}, \ref{tab:w.16}, \ref{tab:w.17}, \ref{tab:w.18}, \ref{tab:w.19}, \ref{tab:w.20} contain, in any case, p-values considerably smaller than the significance level ($\alpha=0.05$). 

% This implies that the null hypothesis corresponding to the equality of the median values of the differences of the two distributions can be rejected, hence assessing the existence of a statistically significant difference among the solution quality generated by analyzed algorithms.

% \item By looking at the Cpu time, one can see that the instances \emph{n80w20.X} have generally lower runtimes than the \emph{n80w200.X} ones. They can then be considered, with respect to the variable neighborhood descent algorithms, simpler (quickier to solve) instances with respect to the latter.

% \end{itemize}

\end{homeworkProblem}
