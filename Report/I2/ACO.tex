%----------------------------------------------------------------------------------------
%	PROBLEM 1
%----------------------------------------------------------------------------------------

% To have just one problem per page, simply put a \clearpage after each problem
\newpage
\begin{homeworkProblem}
\section{Ant Colony Optimization}
\subsection{Problem statement}
Implement two stochastic local search (SLS) algorithms for the traveling salesman problem with time windows (TPSTW), building on top of the perturbative local search methods from the first implementation exercise.
\begin{enumerate}
  \item Run each algorithm 25 times with different random seed on each instance. Instances will be available from http://iridia.ulb.ac.be/˜stuetzle/Teaching/HO/. As termination criterion, for each instance, use the maximum computation time it takes to run a full VND (implemented in the previous exercise) on the same instance and then multiply this time by 1000 (to allow for long enough runs of the SLS algorithms).
 \item Compute the following statistics for each of the two SLS algorithms and each instance:
 \begin{itemize}
   \item Percentage of runs with constraint violations
   \item Mean penalised relative percentage deviation
 \end{itemize}

\item Produce boxplots of penalised relative percentage deviation.
\item Determine, using statistical tests (in this case, the Wilcoxon test), whether there is a statistically significant difference between the quality of the solutions generated by the two algorithms.
\item Measure, for each of the implemented algorithms on 5 instances, the run-time distributions to reach sufficiently high quality solutions (e.g. best-known solutions available at http://iridia.ulb.ac.be/˜manuel/tsptw-instances\#instances).
Measure the run-time distributions across 25 repetitions using a cut-off time of 10 times the termination criterion above.
\item Produce a written report on the implementation exercise:
\begin{itemize}
  \item Please make sure that each implemented SLS algorithm is appropriately described and that the computational results are carefully interpreted. Justify also the choice of the parameter settings and the choice
of the iterative improvement algorithm for the hybrid SLS algorithm.
  \item Present the results as in the previous implementation exercise (tables, boxplots, statistical tests).
  \item Present graphically the results of the analysis of the run-time distributions.
  \item Interpret appropriately the results and make conclusions on the relative performance of the algorithms across all the benchmark instances studied.
\end{itemize}
\end{enumerate}

\subsection{Introduction}
Ant Colony Optimization is an example of population-based stochastic local search method.

The algorithm is inspired by the behavior of the ant species \emph{Iridomyrmex humilis}.

To be more precise, these insects are able, by means of stigmergic communication, to choose the shortest path between their nest and a food source, when given the choice (CITE Deneubourg).

The ants are able to communicate indirectly by deposing a certain quantity of pheromone in the environment that can be sensed by the other ants and that will be used by them to choose the right path.

The convergence to one of the paths will occur as a consequence of the self-reinforcing pheromone deposit mechanism (i.e the more pheromone is deposited, the more ants will follow the pheromone trail deposing even more pheromone).

The Ant Colony Optimization method arise from the implementation of this mechanism to optimization problems (CITE Dorigo), with the following analogies:
\begin{itemize}
\item Ants $\equiv $ Virtual ants
\item Pheromone $\equiv $ Virtual pheromone
\item Environment $\equiv $ Search space 
\end{itemize}

\begin{algorithm}
  \caption{Ant Colony Optimization for TSPTW - Outline}\label{aco}
  \begin{algorithmic}[1]
    \Procedure{ACO}{$\alpha,\beta,\rho,\tau_0,n_{ants},n_{cities},t_{max}$}\Comment{The main procedure}
    \State {\emph{InitalizePheromoneTrail}($\tau_0,n_{cities}$)} \Comment{Set the pheromone values for all the solution components to $\tau_0$}
    \While{\emph{!TerminationCondition}($t_{max}$)}\Comment{The termination condition can be either based on time or solution quality} \label{marker}
      \ForAll {ants}
        \State \emph{ConstructSolution}($\alpha,\beta$)
        \State \emph{LocalSearch}(TO DEFINE)
        \State \emph{UpdatePheromoneTrails}($\rho$)
      \EndFor
    \EndWhile
    \State \textbf{return} solution\Comment{The gcd is b}
    \State
  \EndProcedure
\end{algorithmic}
\end{algorithm}

\begin{algorithm}
  \caption{Pheromone Initialization}\label{init}
  \begin{algorithmic}[1]
    \Procedure{InitalizePheromoneTrail}{$\tau_0,n_{cities}$}\Comment{The main procedure}
      \State $i \gets 0$
      \State $j \gets 0$
      \For{$i < n_{cities}$} 
        \For{$j < i$} 
          \State $\tau_{ij} \gets \tau_0$
          \State $\tau_{ji} \gets \tau_{ij}$
          \State $ j \gets j + 1$  
        \EndFor
        \State $ i \gets i + 1$ 
      \EndFor
    \EndProcedure
\end{algorithmic}
\end{algorithm}

\begin{algorithm}
  \caption{Solution Construction}\label{sol}
  \begin{algorithmic}[1]
    \Procedure{ConstructSolution}{$\alpha,\beta$}\Comment{The main procedure}
      \State \Comment{$s_i$ reprents the $i^{th}$ component of the solution}
      \State $s_0 \gets 0$ \Comment{Every solution starts at the depot}
      \State $s_1 \gets$ \emph{RandomCitySelection}() \Comment{Random choice of the starting city}
      \State $i \gets 1$
      \While{!\emph{SolutionComplete}()}
        \State $k \gets $ \emph{RouletteWheelSelection}() \Comment{ $k$ stochastically chosen according to the probability distribution defined by $p_k$}
        \State $s_i \gets k$ 
        \State $i \gets i+1$
      \EndWhile
    \EndProcedure
\end{algorithmic}
\end{algorithm}

\begin{equation}
p_k = \frac{[\tau_{i-1,k}]^\alpha \cdot [\eta_{i-1,k}]^\beta}{\sum_{j \in N(s_{i-1})} [\tau_{i-1,j}]^\alpha \cdot [\eta_{i-1,j}]^\beta}
\end{equation}  

\begin{algorithm}
  \caption{Pheromone Trails Update}\label{update}
  \begin{algorithmic}[1]
    \Procedure{UpdatePheromoneTrails}{$\rho$}\Comment{The main procedure}
     \State $i \gets 0$
      \State $j \gets 0$
      \For{$i < n_{cities}$} 
        \For{$j < i$} 
          \State $\tau_{ij} \gets (1-\rho)\cdot\tau_{ij}+\Delta_{\tau}$
          \State $\tau_{ji} \gets \tau_{ij}$
          \State $ j \gets j + 1$  
        \EndFor
        \State $ i \gets i + 1$ 
      \EndFor
    \EndProcedure
\end{algorithmic}
\end{algorithm} 

\begin{equation}
  \Delta_{\tau} = DEFINE
\end{equation}

\begin{tikzpicture}[->,>=stealth',shorten >=1pt,auto,node distance=3cm,
                    semithick]
  \tikzstyle{every state}=[fill=blue!20,draw=none,thick]

  \node[initial,state] (Init)                    {};
  \node[state]         (CI) [right of=Init]     {\emph{CI}};
  \node[state]         (CS) [above right of=CI] {\emph{CS}};
  \node[state]         (LS) [below right of=CI] {\emph{LS}};

  \path (Init) edge              node {DET:\emph{IPT}()} (CI)
        (CI) edge              node {DET} (CS)
        (CS) edge [loop above] node {CDET(not \emph{SC})} (CS)
            edge              node {CDET(\emph{SC})} (LS)
        (LS) edge [loop below] node {CDET(not \emph{LO})} (LS)
            edge               node {CDET(\emph{LO})} (CI);
\end{tikzpicture}

\begin{itemize}
  \item \emph{IPT}() $\equiv$ \emph{InitializePheromoneTrail}()
  \item \emph{SC} $\equiv$ \emph{SolutionComplete}()
  \item \emph{LO} $\equiv$ \emph{LocalOptimum}()
\end{itemize}
 
An Iterative Improvement algorithm generally starts from a candidate solution, which can be either generated randomly or using an heuristic, and improves the evaluation of the solution at each step by modifying the solution structure , until a local optimum is reached.

In the previous problem, I considered different kinds of 2-opt neighborhood, and different pivoting rules.

This means that, at each step, a new solution is constructed from the current best by modifying only two solution components (with Transpose,Exchange or Insert operations) and only the first/best improving solution will become the new best soluion.

The main limitation of such kind of algorithms is that they tend to get stuck in solutions that are locally optimum but not globally.

Provided that:
\begin{itemize}
\item A global optimum is optimal with respect to any kind of neighborhood. 
\item A solution that is locally optimal with respect to a neighborhood may not be optimal with respect to other kinds of neighborhood
\end{itemize}
by dynamically changing the neighborhood type an algorithm is able to escape local optima.

This section will analyse the results of the execution of two variable neighborhood descent algorithm, based on the previously analyzed iterative improvement algorithms :
\begin{itemize}
\item \textbf{Standard Variable Neighborhood Descent} (i.e. Changing neighborhood when a local optimum is encoutered, until the neighborhood chain is terminated and going back to the smallest neighborhood every time the local optimum is escaped.)
\item \textbf{Piped Variable Neighborhood Descent} (i.e. Using the locally optimum solution found using one neighborhood type in the chain as the initial solution for the following type.)
\end{itemize}
The same metrics as in \ref{subsec:metric} will be used to evaluate the algorithms.

% \subsection{Experiment results}
% \subsubsection{n80w20.001}
% \begin{center}
% \includegraphics[width=0.6\textwidth,keepaspectratio]{{VND/n80w20.001/n80w20.001-CpuTime}.pdf}
% \captionof{figure}{n80w20.001 - Runtime boxplots for the different variable neighborhood descent algorithms}
% \end{center}

% \begin{center}
% \includegraphics[width=0.6\textwidth,keepaspectratio]{{VND/n80w20.001/n80w20.001-PRPD}.pdf}
% \captionof{figure}{n80w20.001 - PRPD boxplots for the different variable neighborhood descent algorithms}
% \end{center}

% \begin{center}
% \begin{tabular}{|l|l|}
% \hline
% \textbf{Test} & \textbf{P-Value} \\
% \hline
% Tei vs Tie - Standard&3.95591160889952e-18\\
% \hline
% Tei vs Tie - Piped&3.9556885406462e-18\\
% \hline
% Standard vs Piped - Tei&3.95591160889952e-18\\
% \hline
% Standard vs Piped - Tie&3.95591160889952e-18\\
% \hline
% \end{tabular}
% \captionof{table}{n80w20.001 - Results of Wilcoxon paired signed rank test}
% \label{tab:w.11}
% \end{center}

% \subsubsection{n80w20.002}
% \begin{center}
% \includegraphics[width=0.6\textwidth,keepaspectratio]{{VND/n80w20.002/n80w20.002-CpuTime}.pdf}
% \captionof{figure}{n80w20.002 - Runtime boxplots for the different variable neighborhood descent algorithms}
% \end{center}

% \begin{center}
% \includegraphics[width=0.6\textwidth,keepaspectratio]{{VND/n80w20.002/n80w20.002-PRPD}.pdf}
% \captionof{figure}{n80w20.002 - PRPD boxplots for the different variable neighborhood descent algorithms}
% \end{center}

% \begin{center}
% \begin{tabular}{|l|l|}
% \hline
% \textbf{Test} & \textbf{P-Value} \\
% \hline
% Tei vs Tie - Standard&3.9556885406462e-18\\
% \hline
% Tei vs Tie - Piped&3.95591160889952e-18\\
% \hline
% Standard vs Piped - Tei&3.95591160889952e-18\\
% \hline
% Standard vs Piped - Tie&3.95591160889952e-18\\
% \hline
% \end{tabular}
% \captionof{table}{n80w20.002 - Results of Wilcoxon paired signed rank test}
% \label{tab:w.12}
% \end{center}

% \subsubsection{n80w20.003}
% \begin{center}
% \includegraphics[width=0.6\textwidth,keepaspectratio]{{VND/n80w20.003/n80w20.003-CpuTime}.pdf}
% \captionof{figure}{n80w20.003 - Runtime boxplots for the different variable neighborhood descent algorithms}
% \end{center}

% \begin{center}
% \includegraphics[width=0.6\textwidth,keepaspectratio]{{VND/n80w20.003/n80w20.003-PRPD}.pdf}
% \captionof{figure}{n80w20.003 - PRPD boxplots for the different variable neighborhood descent algorithms}
% \end{center}

% \begin{center}
% \begin{tabular}{|l|l|}
% \hline
% \textbf{Test} & \textbf{P-Value} \\
% \hline
% Tei vs Tie - Standard&3.9552424399092e-18\\
% \hline
% Tei vs Tie - Piped&3.95591160889952e-18\\
% \hline
% Standard vs Piped - Tei&3.95591160889952e-18\\
% \hline
% Standard vs Piped - Tie&3.95591160889952e-18\\
% \hline
% \end{tabular}
% \captionof{table}{n80w20.003 - Results of Wilcoxon paired signed rank test}
% \label{tab:w.13}
% \end{center}

% \subsubsection{n80w20.004}
% \begin{center}
% \includegraphics[width=0.6\textwidth,keepaspectratio]{{VND/n80w20.004/n80w20.004-CpuTime}.pdf}
% \captionof{figure}{n80w20.004 - Runtime boxplots for the different variable neighborhood descent algorithms}
% \end{center}

% \begin{center}
% \includegraphics[width=0.6\textwidth,keepaspectratio]{{VND/n80w20.004/n80w20.004-PRPD}.pdf}
% \captionof{figure}{n80w20.004 - PRPD boxplots for the different variable neighborhood descent algorithms}
% \end{center}

% \begin{center}
% \begin{tabular}{|l|l|}
% \hline
% \textbf{Test} & \textbf{P-Value} \\
% \hline
% Tei vs Tie - Standard&3.95591160889952e-18\\
% \hline
% Tei vs Tie - Piped&3.95591160889952e-18\\
% \hline
% Standard vs Piped - Tei&3.95591160889952e-18\\
% \hline
% Standard vs Piped - Tie&3.95591160889952e-18\\
% \hline
% \end{tabular}
% \captionof{table}{n80w20.004 - Results of Wilcoxon paired signed rank test}
% \label{tab:w.14}
% \end{center}

% \subsubsection{n80w20.005}
% \begin{center}
% \includegraphics[width=0.6\textwidth,keepaspectratio]{{VND/n80w20.005/n80w20.005-CpuTime}.pdf}
% \captionof{figure}{n80w20.005 - Runtime boxplots for the different variable neighborhood descent algorithms}
% \end{center}

% \begin{center}
% \includegraphics[width=0.6\textwidth,keepaspectratio]{{VND/n80w20.005/n80w20.005-PRPD}.pdf}
% \captionof{figure}{n80w20.001 - PRPD boxplots for the different variable neighborhood descent algorithms}
% \end{center}

% \begin{center}
% \begin{tabular}{|l|l|}
% \hline
% \textbf{Test} & \textbf{P-Value} \\
% \hline
% Tei vs Tie - Standard&3.95591160889952e-18\\
% \hline
% Tei vs Tie - Piped&3.95591160889952e-18\\
% \hline
% Standard vs Piped - Tei&3.95591160889952e-18\\
% \hline
% Standard vs Piped - Tie&3.95591160889952e-18\\
% \hline
% \end{tabular}
% \captionof{table}{n80w20.005 - Results of Wilcoxon paired signed rank test}
% \label{tab:w.15}
% \end{center}

% \subsubsection{n80w200.001}
% \begin{center}
% \includegraphics[width=0.6\textwidth,keepaspectratio]{{VND/n80w200.001/n80w200.001-CpuTime}.pdf}
% \captionof{figure}{n80w200.001 - Runtime boxplots for the different variable neighborhood descent algorithms}
% \end{center}

% \begin{center}
% \includegraphics[width=0.6\textwidth,keepaspectratio]{{VND/n80w200.001/n80w200.001-PRPD}.pdf}
% \captionof{figure}{n80w200.001 - PRPD boxplots for the different variable neighborhood descent algorithms}
% \end{center}

% \begin{center}
% \begin{tabular}{|l|l|}
% \hline
% \textbf{Test} & \textbf{P-Value} \\
% \hline
% Tei vs Tie - Standard&4.07730530936212e-18\\
% \hline
% Tei vs Tie - Piped&2.92094064174088e-17\\
% \hline
% Standard vs Piped - Tei&2.72456795287507e-16\\
% \hline
% Standard vs Piped - Tie&3.95591160889952e-18\\
% \hline
% \end{tabular}
% \captionof{table}{n80w200.001 - Results of Wilcoxon paired signed rank test}
% \label{tab:w.16}
% \end{center}

% \subsubsection{n80w200.002}
% \begin{center}
% \includegraphics[width=0.6\textwidth,keepaspectratio]{{VND/n80w200.002/n80w200.002-CpuTime}.pdf}
% \captionof{figure}{n80w200.002 - Runtime boxplots for the different variable neighborhood descent algorithms}
% \end{center}

% \begin{center}
% \includegraphics[width=0.6\textwidth,keepaspectratio]{{VND/n80w200.002/n80w200.002-PRPD}.pdf}
% \captionof{figure}{n80w200.002 - PRPD boxplots for the different variable neighborhood descent algorithms}
% \end{center}

% \begin{center}
% \begin{tabular}{|l|l|}
% \hline
% \textbf{Test} & \textbf{P-Value} \\
% \hline
% Tei vs Tie - Standard&3.95591160889952e-18\\
% \hline
% Tei vs Tie - Piped&1.52379449675399e-17\\
% \hline
% Standard vs Piped - Tei&1.74838327736385e-15\\
% \hline
% Standard vs Piped - Tie&3.95591160889952e-18\\
% \hline
% \end{tabular}
% \captionof{table}{n80w200.002 - Results of Wilcoxon paired signed rank test}
% \label{tab:w.17}
% \end{center}

% \subsubsection{n80w200.003}
% \begin{center}
% \includegraphics[width=0.6\textwidth,keepaspectratio]{{VND/n80w200.003/n80w200.003-CpuTime}.pdf}
% \captionof{figure}{n80w200.003 - Runtime boxplots for the different variable neighborhood descent algorithms}
% \end{center}

% \begin{center}
% \includegraphics[width=0.6\textwidth,keepaspectratio]{{VND/n80w200.003/n80w200.003-PRPD}.pdf}
% \captionof{figure}{n80w200.003 - PRPD boxplots for the different variable neighborhood descent algorithms}
% \end{center}

% \begin{center}
% \begin{tabular}{|l|l|}
% \hline
% \textbf{Test} & \textbf{P-Value} \\
% \hline
% Tei vs Tie - Standard&2.04955667109233e-17\\
% \hline
% Tei vs Tie - Piped&2.59611565456869e-17\\
% \hline
% Standard vs Piped - Tei&1.50422804122146e-07\\
% \hline
% Standard vs Piped - Tie&3.95591160889952e-18\\
% \hline
% \end{tabular}
% \captionof{table}{n80w200.003 - Results of Wilcoxon paired signed rank test}
% \label{tab:w.18}
% \end{center}

% \subsubsection{n80w200.004}
% \begin{center}
% \includegraphics[width=0.6\textwidth,keepaspectratio]{{VND/n80w200.004/n80w200.004-CpuTime}.pdf}
% \captionof{figure}{n80w200.004 - Runtime boxplots for the different variable neighborhood descent algorithms}
% \end{center}

% \begin{center}
% \includegraphics[width=0.6\textwidth,keepaspectratio]{{VND/n80w200.004/n80w200.004-PRPD}.pdf}
% \captionof{figure}{n80w200.004 - PRPD boxplots for the different variable neighborhood descent algorithms}
% \end{center}

% \begin{center}
% \begin{tabular}{|l|l|}
% \hline
% \textbf{Test} & \textbf{P-Value} \\
% \hline
% Tei vs Tie - Standard&4.07730530936212e-18\\
% \hline
% Tei vs Tie - Piped&4.29577057320019e-16\\
% \hline
% Standard vs Piped - Tei&5.3075517052254e-11\\
% \hline
% Standard vs Piped - Tie&3.95591160889952e-18\\
% \hline
% \end{tabular}
% \captionof{table}{n80w200.004 - Results of Wilcoxon paired signed rank test}
% \label{tab:w.19}
% \end{center}

% \subsubsection{n80w200.005}
% \begin{center}
% \includegraphics[width=0.6\textwidth,keepaspectratio]{{VND/n80w200.005/n80w200.005-CpuTime}.pdf}
% \captionof{figure}{n80w200.005 - Runtime boxplots for the different variable neighborhood descent algorithms}
% \end{center}

% \begin{center}
% \includegraphics[width=0.6\textwidth,keepaspectratio]{{VND/n80w200.005/n80w200.005-PRPD}.pdf}
% \captionof{figure}{n80w200.001 - PRPD boxplots for the different variable neighborhood descent algorithms}
% \end{center}

% \begin{center}
% \begin{tabular}{|l|l|}
% \hline
% \textbf{Test} & \textbf{P-Value} \\
% \hline
% Tei vs Tie - Standard&1.39380002081336e-17\\
% \hline
% Tei vs Tie - Piped&4.07730530936212e-18\\
% \hline
% Standard vs Piped - Tei&3.72316935219101e-06\\
% \hline
% Standard vs Piped - Tie&3.95591160889952e-18\\
% \hline
% \end{tabular}
% \captionof{table}{n80w200.001 - Results of Wilcoxon paired signed rank test}
% \label{tab:w.20}
% \end{center}

% \subsection{Statistics}
% \subsubsection{Standard-Transpose-Exchange-Insert}
% \begin{center}
% \begin{tabular}{|l|c|l|l|}
% \hline
% \textbf{Instance}& \textbf{\% Infeasible} & $\mathbf{\bar{PRDP}}$ &$\mathbf{\bar{Runtime}}$\\
% \hline
% n80w20.001&0.71&14772.04644164&50.611339\\
% \hline
% n80w20.002&0.88&12888.542&50.727053\\
% \hline
% n80w20.003&0.92&19936.872&50.820348\\
% \hline
% n80w20.004&0.62&17234.94260984&50.049484\\
% \hline
% n80w20.005&0.94&12564.0560428&50.269182\\
% \hline
% n80w200.001&0.28&11212.97389136&49.151249\\
% \hline
% n80w200.002&0.03&629.5853274&51.433949\\
% \hline
% n80w200.003&0.07&1511.56628539&49.082085\\
% \hline
% n80w200.004&0.16&4193.4817209&49.662512\\
% \hline
% n80w200.005&0.01&466.6729061&46.701953\\
% \hline
% \end{tabular}
% \captionof{table}{Statistics summary for variable neighborhood descent algorithm with Transpose-Exchange-Insert neighborhood chain and Standard VND type}
% \label{tab:s.tei}
% \end{center}

% \subsubsection{Standard-Transpose-Insert-Exchange}
% \begin{center}
% \begin{tabular}{|l|c|l|l|}
% \hline
% \textbf{Instance}& \textbf{\% Infeasible} & $\mathbf{\bar{PRDP}}$ &$\mathbf{\bar{Runtime}}$\\
% \hline
% n80w20.001&0.54&10874.77632472&15.268454\\
% \hline
% n80w20.002&0.62&8411.724&15.386641\\
% \hline
% n80w20.003&0.44&7645.295&15.638153\\
% \hline
% n80w20.004&0.39&7153.68881324&15.980347\\
% \hline
% n80w20.005&0.25&3475.2731712&15.55767\\
% \hline
% n80w200.001&0.16&4898.3227617&33.424555\\
% \hline
% n80w200.002&0&11.0430351&32.198479\\
% \hline
% n80w200.003&0.05&1082.1460308&34.345522\\
% \hline
% n80w200.004&0.28&7804.19186258&32.583152\\
% \hline
% n80w200.005&0&10.20227353&34.501294\\
% \hline
% \end{tabular}
% \captionof{table}{Statistics summary for variable neighborhood descent algorithm with Transpose-Insert-Exchange neighborhood chain and Standard VND type}
% \label{tab:s.tie}
% \end{center}

% \subsubsection{Piped-Transpose-Exchange-Insert}
% \begin{center}
% \begin{tabular}{|l|c|l|l|}
% \hline
% \textbf{Instance}& \textbf{\% Infeasible} & $\mathbf{\bar{PRDP}}$ &$\mathbf{\bar{Runtime}}$\\
% \hline
% n80w20.001&0.59&12336.84228578&35.694416\\
% \hline
% n80w20.002&0.94&15603.0142035&36.212393\\
% \hline
% n80w20.003&0.83&19338.924&34.821217\\
% \hline
% n80w20.004&0.55&13170.33921962&36.438959\\
% \hline
% n80w20.005&0.45&6683.336214&36.202891\\
% \hline
% n80w200.001&0.19&5104.3621015&40.772642\\
% \hline
% n80w200.002&0.01&218.8179842&44.241593\\
% \hline
% n80w200.003&0.06&2584.90674231&44.725066\\
% \hline
% n80w200.004&0.17&3430.64506042&43.760992\\
% \hline
% n80w200.005&0.02&693.0136326&42.646023\\
% \hline
% \end{tabular}
% \captionof{table}{Statistics summary for variable neighborhood descent algorithm with Transpose-Exchange-Insert neighborhood chain and Piped VND type}
% \label{tab:p.tei}
% \end{center}

% \subsubsection{Piped-Transpose-Insert-Exchange}
% \begin{center}
% \begin{tabular}{|l|c|l|l|}
% \hline
% \textbf{Instance}& \textbf{\% Infeasible} & $\mathbf{\bar{PRDP}}$ &$\mathbf{\bar{Runtime}}$\\
% \hline
% n80w20.001&0.68&16393.81210394&24.788225\\
% \hline
% n80w20.002&0.81&11667.654&25.902581\\
% \hline
% n80w20.003&0.84&20537.669&26.442309\\
% \hline
% n80w20.004&0.46&8779.79314651&26.424231\\
% \hline
% n80w20.005&0.24&3876.3661498&26.511156\\
% \hline
% n80w200.001&0.21&5917.0312803&52.302366\\
% \hline
% n80w200.002&0.01&626.0983757&56.238843\\
% \hline
% n80w200.003&0.04&867.92563269&58.498874\\
% \hline
% n80w200.004&0.28&6281.58944822&55.867038\\
% \hline
% n80w200.005&0.01&236.8657243&58.331595\\
% \hline
% \end{tabular}
% \captionof{table}{Statistics summary for variable neighborhood descent algorithm with Transpose-Insert-Exchange neighborhood chain and Piped VND type}
% \label{tab:p.tie}
% \end{center}

\subsection{Results discussion}
By looking at tables \ref{tab:s.tei}, \ref{tab:s.tie}, \ref{tab:p.tei}, \ref{tab:p.tie} one can see that:
\begin{itemize}

\item For some instances (e.g. $n80w20.002$,$n80w20.003$) the algorithm are not able to converge to a feasible solution, as shown in the corresponding boxplots, since the PRPD distribution is centered around 12000-15000, thus indicating the presence of at least 1 constraint violations in most of the cases.

\item For some other instances (e.g. $n80w20.004$,$n80w20.005$) the algorithms are able to converge to feasible solutions and to the best-known one, but having a right-skewed distribution towards higher values of PRPD.

\item For the remaining instances, except for some outlier values, the algorithms are able to converge to the best-known solution in most of the runs , even though the average PRPD is not closer to 0. This is due to the fact that the mean of a distribution is sensible to outliers and the penalisation for a constraint violations is extremely high when compared to the mean value.
      
\item The algorithm ordering in terms of runtimes is $s.tie < p.tie < p.tei < s.tei$ for the  $n80w20.X$ instances while $s.tie < p.tei < s.tei < p.tie$ for $n80w200.X$ ones. The choice to explore the Insert Neighborhood before the Exchange one allows to reduce the computation time for the $n80w20.X$ instances, with a similar solution quality.

\item The algorithms are more effective on the $n80w200.X$ instances then the $n80w20.X$ once, since they have a lower percentage of infeasible runs and a lower PRPD.

\item The standard variable neighborhood descent with Transpose-Insert-Exchange neighborhood chain (s.tie) outperforms all the other algorithms in terms of solution quality and runtime.

\item Tables \ref{tab:w.11}, \ref{tab:w.12}, \ref{tab:w.13}, \ref{tab:w.14}, \ref{tab:w.15}, \ref{tab:w.16}, \ref{tab:w.17}, \ref{tab:w.18}, \ref{tab:w.19}, \ref{tab:w.20} contain, in any case, p-values considerably smaller than the significance level ($\alpha=0.05$). 

This implies that the null hypothesis corresponding to the equality of the median values of the differences of the two distributions can be rejected, hence assessing the existence of a statistically significant difference among the solution quality generated by analyzed algorithms.

\item By looking at the Cpu time, one can see that the instances \emph{n80w20.X} have generally lower runtimes than the \emph{n80w200.X} ones. They can then be considered, with respect to the variable neighborhood descent algorithms, simpler (quickier to solve) instances with respect to the latter.

\end{itemize}

\end{homeworkProblem}
